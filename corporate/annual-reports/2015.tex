\documentclass[letterpaper]{report}
\usepackage{parskip}
\usepackage{hyperref}

\hypersetup{
  colorlinks   = true,
  urlcolor     = blue,
  linkcolor    = blue,
}

\begin{document}

\title{Software in the Public Interest, Inc.\\
2015 Annual Report}
\date{July 1, 2015}

\maketitle

To the membership, board and friends of Software in the Public Interest, Inc:

As mandated by Article 8 of the SPI Bylaws, I respectfully submit this annual
report on the activities of Software in the Public Interest, Inc. and extend my
thanks to all of those who contributed to the mission of SPI in the past year.

  \emph{-- Martin Michlmayr, SPI Secretary}

\newpage

\tableofcontents

\newpage

\chapter{President's Welcome}
\label{sec:president}

  \emph{-- Bdale Garbee, SPI President}

\chapter{Committee Reports}
\section{Membership Committee}

\subsection{Statistics}

At the time of writing (July 1) the current membership status is:

\begin{tabular}{ | c | r | }
\hline
NC Applicants Pending Email Approval	& 255\\
NC Members				& 512\\
Contrib Membership Applications		& 8\\
Contrib Members				& 518\\
Application Managers			& 9\\
\hline
\end{tabular}

On July 1, 2014 we had 504 contributing and 455 non-contributing members.
On July 1, 2015 there were 518 contributing members and 512 non-contributing
members.


\chapter{Board Report}
\section{Board Members}

Board members as of July 1, 2014:

\begin{itemize}
\item Bdale Garbee (President)
\item Joerg Jaspert (Vice President)
\item Jonathan McDowell (Secretary)
\item Michael Schultheiss (Treasurer)
\item Clint Adams
\item Robert Brockway
\item Joshua D. Drake
\item Martin Michlmayr
\item Martin Zobel-Helas
\end{itemize}

Board members as of June 30, 2015:

\begin{itemize}
\item Bdale Garbee (President)
\item Joerg Jaspert (Vice President)
\item Martin Michlmayr (Secretary)
\item Michael Schultheiss (Treasurer)
\item Robert Brockway
\item Joshua D. Drake
\item Jonathan McDowell
\item Gregers Petersen
\item Martin Zobel-Helas
\end{itemize}

Advisors to the board as of June 30, 2015:

\begin{itemize}
\item Software Freedom Law Center (SFLC), legal counsel
\item Neil McGovern, Debian Project representative
\item Robert Treat, PostgreSQL Project representative
\end{itemize}

\section{Board Changes}

Changes that occurred during the year:

\begin{itemize}
\item The terms for Clint Adams, Robert Brockway and Martin Michlmayr
expired in July 2014.  Robert and Martin sought, and obtained,
re-election.  We thank Clint for his work on the board.  Gregers
Petersen joined the board as part of the same election.
\item On August 14, 2014 the board voted to extend the term of the
current President, Vice President and Treasurer by a further year.
These were:
\begin{itemize}
\item President: Bdale Garbee
\item Vice President: Joerg Jaspert
\item Treasurer: Michael Schultheiss
\end{itemize}
The board appointed Martin Michlmayr as new Secretary.
\end{itemize}

\section{Elections}

One board membership election was conducted in July 2014.  There were 3
board seats up for election.  Nominations were received from Robert
Brockway, Steve Langasek, Ben Longbons, Martin Michlmayr, Gregers
Petersen, and Trevor Walkley.  Robert Brockway, Martin Michlmayr, and
Gregers Petersen were elected to the board.

\chapter{Treasury Report}

This report uses a cash-based method of accounting, recording donations when
deposited (not when the check was written or received by us) and recording
expenses when sent or scheduled for payment (not when incurred).

\section{Income Statement}

This covers the Period July 1, 2014 -- June 30, 2015

\begin{verbatim}
 Income
\end{verbatim}

\section{Balance Sheet}

\begin{verbatim}
As of June 30, 2015
\end{verbatim}

(all sums in US dollars)


\chapter{Member Project Reports}

\section{New Associated Projects}

We have continued to see a reasonable level of interest from projects who wish
to become associated with SPI.  Over the past year one project has had
a successful resolution proposed for it to be invited to come under the SPI
umbrella as an Associated Project.


\subsection{The Mana World}

The Mana World (TMW) is a serious effort to create an innovative free and
open source MMORPG (Massively Multiplayer Online Role-Playing Game).  TMW
uses 2D graphics and aims to create a large and diverse interactive world.
It is licensed under the GPL.

\section{Updates from Member Projects}

\subsection{0 A.D.}

0 A.D. (pronounced "zero ey-dee") is a cross-platform, real-time strategy
(RTS) game of ancient warfare. It's a historically-based war/economy game,
in which the player must lead an ancient civilization, gather resources
from the map, and raise a military force to conquer enemy factions. 0 A.D.
is open source software licensed under the GPL, and its art and sound
assets are licensed under CC BY-SA. It is developed by Wildfire Games, a
global community of game developers.

Between July 1 2014 to June 30 2015, we put out two alpha releases, each
available for Windows, OS X, Linux, and BSD, including long awaited
features as triggers, nomad maps, fogging, units on walls, a technology
tree, and new Seleucid structures. We were also able to upgrade the
codebase to use new technologies like C++11, SpiderMonkey 31, and a new
auto-builder, thanks to the new server which is funded thanks to our
donations via SPI. Apart from that, many performance improvements were
included and recently the new path-finder was merged.

{\em Submitted by Aviv Sharon}

\subsection{Chakra}

Chakra is a GNU/Linux distribution with an emphasis on KDE and Qt
technologies that focuses on simplicity from a technical standpoint and
free software.  In the past twelve months the Chakra community has made
three releases; 2014.09, 2014.11 and 2015.03.  We have also created two
additional virtual machines on our server for our unofficial Italian- and
Spanish-speaking communities to host their websites.  Last but not least,
we also gained an additional sponsor; JetBrains s.r.o.

{\em Submitted by H W ``totte'' Tovetjärnfor}

\subsection{Debian}

The last twelve months have been a busy one for Debian, with
preparations for Debian 8.0, codenamed Jessie, reaching its final stages,
leading to a release on April 25, 2015. Debian 8.0 released with two
new architectures and a whole host of updated packages.

Additionally, efforts were started to increase the reliability and
verifiability of packages in Debian with source-only uploads, reproducible
builds and the removal of GPG keys less than 2048 bits from the keyring.

With the wider community, Debian again participated in Google Summer of
Code and Outreachy, as well as five sprints and the annual DebConf
gathering~--this time held in Portland, USA.

{\em Submitted by Neil McGovern}

\subsection{FFmpeg}

FFmpeg is a complete, cross-platform solution to record, convert and stream
audio and video. It is used as the platform foundation of many projects
dealing with multimedia, both open source and proprietary, and used
extensively by several multimedia web-based multimedia conversion and
processing services.

In the last twelve months FFmpeg delivered three main releases (2.4, 2.5,
2.6) and several security updates of old releases. A complete list of
changes can be
\href{http://git.videolan.org/?p=ffmpeg.git;a=blob_plain;f=Changelog;hb=HEAD}{found
in the changelog}.

In the last year FFmpeg participated into several development programs,
including OPW (now Outreachy) and Google Summer of Code.

{\em Submitted by Stefano Sabatini}

\subsection{FreedomBox}

FreedomBox Foundation has made substantial progress in the building of both
its educational program and technical demonstration materials in the past
year.  The FreedomBox server software stack will be having its 0.4 release
shortly.  We expect our demonstration software stack to be publicly shown
in a fully functional product line in late 2015 or early 2016.

The Foundation is also developing new materials for privacy education
around free technology; we will be announcing new initiatives in this
direction through 2015.

The Foundation plans to seek independent 501(c)(3) determination in the
near future.

{\em Submitted by Eben Moglen}

\subsection{LibreOffice}

LibreOffice follows a time-based release model to the benefit of not only
our end-users, but also our developers. New features are released to the
public in due time, and improvements are made available on a regular basis.
In 2014, The Document Foundation announced two major releases of
LibreOffice---LibreOffice 4.2 on January 30 and LibreOffice 4.3 on July 30.
15 minor releases have been made available as well. The LibreOffice Impress
Remote for iPhone and iPad was announced on March 2nd, for a total of 18
announcements in 12 months (on average, one announcement every 2.8 weeks or
20 days, which represents a significant achievement for a community-based
project). Developers started working on LibreOffice 4.4, and QA volunteers
organized two bug hunting sessions: the first in November after the release
of the first beta, and the second in December after the availability of the
first release candidate.

In addition to LibreOffice, there have been several announcements of
related products such as LibreOffice Portable, which allows to run
LibreOffice from USB key, and the LibreOffice viewer for Android. Third
parties have also announced solutions based on LibreOffice such as CloudOn
with the iPad editor, and RollApp with the virtualization technology which
brings LibreOffice on iOS, Android, Chrome OS and now Firefox OS. At the
end of 2014, the estimated user base of LibreOffice is exceeding 80 million
users worldwide according to the number of Windows and OS X users pinging
for updates, plus Linux users updating their software from repositories.

For more information on LibreOffice and The Document Foundation, see the
full
\href{https://wiki.documentfoundation.org/File:TDF2014AnnualReport.pdf}{2014
annual report}.

{\em Submitted by Sophie Gautier}


\appendix
\chapter{About SPI}

SPI is a non-profit organization which was founded to help organizations
develop and distribute open hardware and software. We encourage programmers
to use the GNU General Public License or other licenses that allow free
redistribution and use of software, and hardware developers to distribute
documentation that will allow device drivers to be written for their product.

SPI was incorporated as a non-profit organization on June 16, 1997 in the state
of New York. Since then, it has become an umbrella organization for projects
from the community.

In 1999, the Internal Revenue Service (IRS) of the United States government
determined that under section 501 (a) of the Internal Revenue Code SPI
qualifies for 501 (c) (3) (non-profit organization) status under section 509
(a) (1) and 170 (b) (1) (A) (vi). This means that donations made to SPI and its
supported projects should be tax deductible for the American donor.

\end{document}
% Keep this at the bottom, thanks.
% Local Variables:
% TeX-master: "report"
% End:
