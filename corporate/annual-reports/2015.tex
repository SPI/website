\documentclass[letterpaper]{report}
\usepackage{parskip}
\usepackage{hyperref}

\hypersetup{
  colorlinks   = true,
  urlcolor     = blue,
  linkcolor    = blue,
}

\begin{document}

\title{Software in the Public Interest, Inc.\\
2015 Annual Report}
\date{July 1, 2016}

\maketitle

To the membership, board and friends of Software in the Public Interest, Inc:

As mandated by Article 8 of the SPI Bylaws, I respectfully submit this annual
report on the activities of Software in the Public Interest, Inc. and extend my
thanks to all of those who contributed to the mission of SPI in the past year.

  \emph{-- Martin Michlmayr, SPI Secretary}

\newpage

\tableofcontents

\newpage

\chapter{President's Welcome}
\label{sec:president}

  \emph{-- Bdale Garbee, SPI President}

\chapter{Committee Reports}
\section{Membership Committee}

\subsection{Statistics}

On January 1, 2015 we had 512 contributing and 501 non-contributing members.
On December 31, 2015 there were 517 contributing members and 520 non-contributing
members.


\chapter{Board Report}
\section{Board Members}

Board members as of January 1, 2015:

\begin{itemize}
\item Bdale Garbee (President)
\item Joerg Jaspert (Vice President)
\item Martin Michlmayr (Secretary)
\item Michael Schultheiss (Treasurer)
\item Robert Brockway
\item Joshua D. Drake
\item Jonathan McDowell
\item Gregers Petersen
\item Martin Zobel-Helas
\end{itemize}

Board members as of December 31, 2015:

\begin{itemize}
\item Bdale Garbee (President)
\item Joerg Jaspert (Vice President)
\item Martin Michlmayr (Secretary)
\item Michael Schultheiss (Treasurer)
\item Robert Brockway
\item Joshua D. Drake
\item Dimitri John Ledkov
\item Gregers Petersen
\item Martin Zobel-Helas
\end{itemize}

Advisors to the board as of December 31, 2015:

\begin{itemize}
\item Software Freedom Law Center (SFLC), legal counsel
\item Neil McGovern, Debian Project representative
\item Robert Treat, PostgreSQL Project representative
\end{itemize}

\section{Board Changes}

Changes that occurred during the year:

\begin{itemize} \item The terms for Jonathan McDowell and Michael
Schultheiss expired in July 2015.  Michael sought, and obtained,
re-election.  We thank Jonathan for his work on the board.  Dimitri John
Ledkov joined the board as part of the same election.

\item On September 10, 2015 the board voted to extend the term of the
current officers by a further year. These were:
\begin{itemize}
\item President: Bdale Garbee
\item Vice President: Joerg Jaspert
\item Secretary: Martin Michlmayr
\item Treasurer: Michael Schultheiss
\end{itemize}
\end{itemize}

\section{Elections}

One board membership election was conducted in July 2015.  There were 2
board seats up for election.  Nominations were received from Dimitri
John Ledkov and Michael Schultheiss.  Since there were 2 nominations for
2 board seats, no vote was required and Dimitri John Ledkov and Michael
Schultheiss were elected for a 3 year term.

\chapter{Treasury Report}

This report uses a cash-based method of accounting, recording donations when
deposited (not when the check was written or received by us) and recording
expenses when sent or scheduled for payment (not when incurred).

\section{Income Statement}

This covers the Period January 1, 2015 -- December 31, 2015

\begin{verbatim}
\end{verbatim}

\section{Balance Sheet}

\begin{verbatim}
\end{verbatim}

(all sums in US dollars)


\chapter{Member Project Reports}

\section{New Associated Projects}

We have continued to see a reasonable level of interest from projects who
wish to become associated with SPI.  However, no new project joined the
SPI umbrella as an associated project in 2015.


\section{Updates from Associated Projects}

\subsection{GNU TeXmacs}

The TeXmacs project continued to working on the release of TeXmacs 2.1,
which involves a lot of code stabilization and adapting it to new
technology, such as retina displays.  New features include the ability to
use arbitrary Type 1 fonts for mathematical typesetting, as well as many
improvements for the presentation mode, and better support for animations.

{\em Submitted by Joris van der Hoeven}

\subsection{OFTC}

OFTC has been in a mostly steady state over the past years, i.e. the IRC
network was up and running, without much newsworthy things happening.
Lately, we've started modernizing our infrastructure, moving much of the
config management to Puppet, replacing the certificates issued by the aged
SPI certificate authority by modern Let's Encrypt certificates, and
generally looking into making the network more resilient against single
points of failure.

{\em Submitted by Christoph Berg}

\subsection{The Mana World}

The Mana World has been able to gain more independence as a project in
2015. The community has raised enough money to pay its costs. The overall
costs were reduced by SPI taking over as registrar and DNS and Let's
Encrypt issuing SSLs. TMWC (The Mana World Committee) has formed an board
to manage TMW as a NGO/Brand, pay bills, inter-project coordination, etc.
We've had steady contributions and major progress towards project
development goals. We pushed multiple versions with numerous outstanding
bugs/requests, increased our amount of content while maintaining or
reducing the code base, continuing to train developers and adding new
developers to the project as well. TMW also made a serious and successful
brand push in 2015; Tying all various social platforms/mediums together,
added responsive web design, updated website to use schema.org, focused on
cross-device usability. Socially the project has become progressive; Adding
non-binary to the list of playable genders in-game, active members of staff
trying to moderate a massively multiplayer game (MMO) as a safe space (free
of hate speech or oppressive behaviors). Active members include a wide and
diverse array of human beings on the planet and manage to co-exist.

{\em Submitted by Edward Pasek}


\appendix
\chapter{About SPI}

SPI is a non-profit organization which was founded to help organizations
develop and distribute open hardware and software. We encourage programmers
to use the GNU General Public License or other licenses that allow free
redistribution and use of software, and hardware developers to distribute
documentation that will allow device drivers to be written for their product.

SPI was incorporated as a non-profit organization on June 16, 1997 in the state
of New York. Since then, it has become an umbrella organization for projects
from the community.

In 1999, the Internal Revenue Service (IRS) of the United States government
determined that under section 501 (a) of the Internal Revenue Code SPI
qualifies for 501 (c) (3) (non-profit organization) status under section 509
(a) (1) and 170 (b) (1) (A) (vi). This means that donations made to SPI and its
supported projects should be tax deductible for the American donor.

\end{document}
% Keep this at the bottom, thanks.
% Local Variables:
% TeX-master: "report"
% End:
