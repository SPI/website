\documentclass[letterpaper]{report}
\usepackage[utf8]{inputenc}
\usepackage{parskip}
\usepackage{hyperref}

\hypersetup{
  colorlinks   = true,
  urlcolor     = blue,
  linkcolor    = blue,
  pdfinfo = {
    Title = {SPI Annual Report 2015},
    Author = {Software in the Public Interest, Inc.},
    Keywords = {SPI, free software, open source, FOSS, annual report, charity, non-profit, 501c3},
  }
}

\begin{document}

\title{Software in the Public Interest, Inc.\\
2015 Annual Report}
\date{July 1, 2016}

\maketitle

To the membership, board and friends of Software in the Public Interest, Inc:

As mandated by Article 8 of the SPI Bylaws, I respectfully submit this annual
report on the activities of Software in the Public Interest, Inc. and extend my
thanks to all of those who contributed to the mission of SPI in the past year.

  \emph{-- Martin Michlmayr, SPI Secretary}

\newpage

\tableofcontents

\newpage

\chapter{President's Welcome}
\label{sec:president}

  \emph{-- Bdale Garbee, SPI President}

\chapter{Committee Reports}
\section{Membership Committee}

\subsection{Statistics}

On January 1, 2015 we had 512 contributing and 501 non-contributing members.
On December 31, 2015 there were 517 contributing members and 520 non-contributing
members.


\chapter{Board Report}
\section{Board Members}

Board members as of January 1, 2015:

\begin{itemize}
\item Bdale Garbee (President)
\item Joerg Jaspert (Vice President)
\item Martin Michlmayr (Secretary)
\item Michael Schultheiss (Treasurer)
\item Robert Brockway
\item Joshua D. Drake
\item Jonathan McDowell
\item Gregers Petersen
\item Martin Zobel-Helas
\end{itemize}

Board members as of December 31, 2015:

\begin{itemize}
\item Bdale Garbee (President)
\item Joerg Jaspert (Vice President)
\item Martin Michlmayr (Secretary)
\item Michael Schultheiss (Treasurer)
\item Robert Brockway
\item Joshua D. Drake
\item Dimitri John Ledkov
\item Gregers Petersen
\item Martin Zobel-Helas
\end{itemize}

Advisors to the board as of December 31, 2015:

\begin{itemize}
\item Software Freedom Law Center (SFLC), legal counsel
\item Neil McGovern, Debian Project representative
\item Robert Treat, PostgreSQL Project representative
\end{itemize}

\section{Board Changes}

Changes that occurred during the year:

\begin{itemize} \item The terms for Jonathan McDowell and Michael
Schultheiss expired in July 2015.  Michael sought, and obtained,
re-election.  We thank Jonathan for his work on the board.  Dimitri John
Ledkov joined the board as part of the same election.

\item On September 10, 2015 the board voted to extend the term of the
current officers by a further year. These were:
\begin{itemize}
\item President: Bdale Garbee
\item Vice President: Joerg Jaspert
\item Secretary: Martin Michlmayr
\item Treasurer: Michael Schultheiss
\end{itemize}
\end{itemize}

\section{Elections}

One board membership election was conducted in July 2015.  There were 2
board seats up for election.  Nominations were received from Dimitri
John Ledkov and Michael Schultheiss.  Since there were 2 nominations for
2 board seats, no vote was required and Dimitri John Ledkov and Michael
Schultheiss were elected for a 3 year term.

\chapter{Treasury Report}

This report uses a cash-based method of accounting, recording donations when
deposited (not when the check was written or received by us) and recording
expenses when sent or scheduled for payment (not when incurred).

\section{Income Statement}

This covers the Period January 1, 2015 -- December 31, 2015

\begin{verbatim}
\end{verbatim}

\section{Balance Sheet}

\begin{verbatim}
\end{verbatim}

(all sums in US dollars)


\chapter{Member Project Reports}

\section{New Associated Projects}

We have continued to see a reasonable level of interest from projects who
wish to become associated with SPI.  However, no new project joined the
SPI umbrella as an associated project in 2015.


\section{Updates from Associated Projects}

\subsection{Chakra}

Chakra is a GNU/Linux distribution with an emphasis on KDE and Qt
technologies that focuses on simplicity from a technical standpoint and
free software.  Chakra has seen several improvements over the past year,
such as new contributors joining our community, two releases having been
made, a switch from KDE SC 4 to Plasma 5, a switch from Tribe to Calamares
as the installation software and finally adding UEFI support.

{\em Submitted by H W ``totte'' Tovetjärnfor}

\subsection{Debian}

Debian remains attached to its Social Contract and places users and free
software community interests first in its priorities: Debian 8 `Jessie'
was released with more than 43,000 packages and 10 computer architectures.
Additionally, the project continued its efforts to run the LTS program for
extended security support.  To celebrate Jessie's release, many parties
were organised all over the world.

To protect its work, the Debian project announced the Copyright
Aggregation Project with Software Freedom Conservancy.  On behalf of the
Debian community, Conservancy can now safeguard the long term interests of
Debian and its commitment to software freedom.

Some Debian contributors were awarded in 2015.  Stefano Zacchiroli, a
former Debian Project Leader, received an O'Reilly Open Source Award at
OSCON for his contributions to Debian and the FOSS community.  And Holger
Levsen and Jérémy Bobbio were funded by The Linux Foundation's Core
Infrastructure Initiative to advance their Debian work on reproducible
builds.

On 16th August 2015, the Debian project celebrated its 22nd anniversary.
This is a new milestone for the project and makes it one of the oldest
Free and Open Source GNU/Linux distributions.  Its annual DebConf
gathering was held in Heidelberg and set a new attendance record with more
than 550 people.  During DebConf, the new Debian Outreach team welcomed 17
students selected to be part of the Google Summer of Code, two of them
being Outreachy applicants.

The year closed with very sad news for the Debian community.  On December
28th, Debian lost the founder of its community and project, Ian Murdock.
The Debian community is thankful to Ian and will protect his legacy by
widening Debian's community and reach and standing up for free software.

{\em Submitted by Mehdi Dogguy}

\subsection{Drizzle}

While the Drizzle server is no longer actively developed, the Drizzle JDBC
driver was actively maintained.  This driver has a use use case as a
permissively licensed JDBC driver supporting MySQL.

{\em Submitted by Henrik Ingo}

\subsection{FFmpeg}

FFmpeg is a complete, cross-platform solution to record, convert and
stream audio and video. It is used as the platform foundation of many
projects dealing with multimedia, both open source and proprietary, and
used extensively by several multimedia web-based multimedia conversion and
processing services.

In 2015 FFmpeg delivered three main releases (2.6, 2.7, 2.8) and several
security updates of old releases. A complete list of changes can be
\href{http://git.videolan.org/?p=ffmpeg.git;a=blob_plain;f=Changelog;hb=HEAD}{found
in the changelog}.

FFmpeg participated as a member of Google Summer of Code 2015.

{\em Submitted by Stefano Sabatini}

\subsection{GNU TeXmacs}

The TeXmacs project continued to working on the release of TeXmacs 2.1,
which involves a lot of code stabilization and adapting it to new
technology, such as retina displays.  New features include the ability to
use arbitrary Type 1 fonts for mathematical typesetting, as well as many
improvements for the presentation mode, and better support for animations.

{\em Submitted by Joris van der Hoeven}

\subsection{Jenkins}

2015 has been a great year for the Jenkins project. We've reached the
milestone of 10 years, 1000 plugins, and 100K installations, and we also
kicked Jenkins 2.0 in motion, which is a first major version number
increase in this project.

{\em Submitted by Kohsuke Kawaguchi}

\subsection{LibreOffice}

The Document Foundation (TDF) welcomed the City of Munich and RusBITech,
which is based in Moscow, as Advisory Board members, while the
Documentation Liberation Project (DLP) celebrated its first birthday in
April. In July, the Open Document Format version 1.2 was published as
International Standard 26300:2015 by ISO/IEC.

LibreOffice became 5 years old in 2015, and took major steps forward with
two feature-packed releases: LibreOffice 4.4 in January and LibreOffice 5.0
in August, regular updates, and the release of LibreOffice Viewer for
Android devices.

In September, TDF organised the LibreOffice Conference in Aarhus, Denmark,
to bring together developers and users to work on code, share ideas and
discuss the future of the project. The conference was also an opportunity
for TDF board and staff to meet in person.

{\em Submitted by Sophie Gautier}

\subsection{OFTC}

OFTC has been in a mostly steady state over the past years, i.e. the IRC
network was up and running, without much newsworthy things happening.
Lately, we've started modernizing our infrastructure, moving much of the
config management to Puppet, replacing the certificates issued by the aged
SPI certificate authority by modern Let's Encrypt certificates, and
generally looking into making the network more resilient against single
points of failure.

{\em Submitted by Christoph Berg}

\subsection{Privoxy}

In 2015 the Privoxy project published one stable releases (3.0.23) which
did not contain any major new features.

Compared to previous years development has slowed down a bit.  Due to
various issues with the hosting service a fair amount of developer time
was spent on finding an alternative that would allow Privoxy developers to
concentrate on development again.

{\em Submitted by Fabian Keil}

\subsection{The Mana World}

The Mana World has been able to gain more independence as a project in
2015. The community has raised enough money to pay its costs. The overall
costs were reduced by SPI taking over as registrar and DNS and Let's
Encrypt issuing SSLs. TMWC (The Mana World Committee) has formed an board
to manage TMW as a NGO/Brand, pay bills, inter-project coordination, etc.
We've had steady contributions and major progress towards project
development goals. We pushed multiple versions with numerous outstanding
bugs/requests, increased our amount of content while maintaining or
reducing the code base, continuing to train developers and adding new
developers to the project as well. TMW also made a serious and successful
brand push in 2015; Tying all various social platforms/mediums together,
added responsive web design, updated website to use schema.org, focused on
cross-device usability. Socially the project has become progressive; Adding
non-binary to the list of playable genders in-game, active members of staff
trying to moderate a massively multiplayer game (MMO) as a safe space (free
of hate speech or oppressive behaviors). Active members include a wide and
diverse array of human beings on the planet and manage to co-exist.

{\em Submitted by Edward Pasek}


\appendix
\chapter{About SPI}

SPI is a non-profit organization which was founded to help organizations
develop and distribute open hardware and software. We encourage programmers
to use the GNU General Public License or other licenses that allow free
redistribution and use of software, and hardware developers to distribute
documentation that will allow device drivers to be written for their product.

SPI was incorporated as a non-profit organization on June 16, 1997 in the state
of New York. Since then, it has become an umbrella organization for projects
from the community.

In 1999, the Internal Revenue Service (IRS) of the United States government
determined that under section 501 (a) of the Internal Revenue Code SPI
qualifies for 501 (c) (3) (non-profit organization) status under section 509
(a) (1) and 170 (b) (1) (A) (vi). This means that donations made to SPI and its
supported projects should be tax deductible for the American donor.

\end{document}
% Keep this at the bottom, thanks.
% Local Variables:
% TeX-master: "report"
% End:
