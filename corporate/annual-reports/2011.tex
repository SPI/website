\documentclass[letterpaper]{report}
\usepackage{hyperref}

\begin{document}

\title{Software in the Public Interest, Inc.\\
2011 Annual Report}
\date{1st July 2011}

\maketitle

To the membership, board and friends of Software in the Public Interest, Inc:

As mandated by Article 8 of the SPI Bylaws, I respectfully submit this annual
report on the activities of Software in the Public Interest, Inc. and extend my
thanks to all of those who contributed to the mission of SPI in the past year.

  \emph{-- Jonathan McDowell, SPI Secretary}

\newpage

\tableofcontents

\newpage

\chapter{President's Welcome}
\label{sec:president}

In the time that has passed since I originally agreed to serve as President
several years ago, SPI has made significant and enduring progress resolving
long-standing procedural problems that once impeded our ability to meet the
basic service expectations of our associated projects.

These accomplishments are of course not my own, but the collective result of
work contributed by a dedicated core group of volunteers including the rest of
our officers and board.

A measure of our success is the steady stream of new SPI associated projects in
recent years, including those we invited to join us in the last year listed
later in this report.

As we look forward, I believe SPI needs to remain focused on our core services.
Financial transaction processing and other basic processes aren't glamorous,
but our primary role in the Free Software ecosystem is handling those details
on behalf of our associated projects, so that their energy can go to producing
and distributing great software!

We continue to lament long-standing issues with our bylaws that deserve to be
addressed.  This is also not glamorous work, but I hope we can make time in the
next year to finally work through updating them.

Thank you to everyone, particularly our contributing members, whose
ongoing contributions of time and attention make SPI possible!

  \emph{-- Bdale Garbee, SPI President}

\chapter{Committee Reports}
\section{Membership Committee}

The membership committee for the year comprised of Luk Claes, Neil McGovern and
Michael Schultheiss.

\subsection{Statistics}

At the time of writing (July 7th) the current membership status is:

\begin{tabular}{ | c | r | }
\hline
NC Applicants Pending Email Approval	& 72 \\
NC Members				& 436 \\
Contrib Membership Applications		& 15 \\
Contrib Members				& 449 \\
Application Managers			& 4 \\
\hline
\end{tabular}

On 1st July 2010 we had 437 contributing and 404 non-contributing members. On
1st July 2011 there were 445 contributing members and 436 non-contributing
members.

\chapter{Board Report}
\section{Board Members}

Board members as of July 1st, 2010:

\begin{itemize}
\item Bdale Garbee (President)
\item Joerg Jaspert (Vice President)
\item Jonathan McDowell (Secretary)
\item Michael Schultheiss (Treasurer)
\item Luk Claes
\item Joshua D. Drake
\item David Graham
\item Jimmy Kaplowitz
\item Martin Zobel-Helas
\end{itemize}

Board members as of June 30th, 2011:

\begin{itemize}
\item Bdale Garbee (President)
\item Joerg Jaspert (Vice President)
\item Jonathan McDowell (Secretary)
\item Michael Schultheiss (Treasurer)
\item Joshua D. Drake
\item David Graham
\item Jimmy Kaplowitz
\item Martin Zobel-Helas
\end{itemize}

Advisors to the board as of June 30th, 2011:

\begin{itemize}
\item Gregory Pomerantz, legal counsel
\item Software Freedom Law Center (SFLC), legal counsel
\item Stefano Zacchiroli, Debian Project representative
\item Robert Treat, PostgreSQL Project representative
\end{itemize}

\section{Board Changes}

Changes that occurred during the year:

\begin{itemize}
\item The terms for Luk Claes, Joshua D. Drake. Bdale Garbee, Joerg Jaspert and
Martin Zobel-Helas expired in July 2010. All except Luk sought, and obtained,
re-election.
\item On July 14th, 2011 the board voted to extend the term of the current
officers by a further year. These were:
\begin{itemize}
\item President: Bdale Garbee
\item Vice President: Joerg Jaspert
\item Secretary: Jonathan McDowell
\item Treasurer: Michael Schultheiss
\end{itemize}
\end{itemize}

\section{Elections}

One board membership election was conducted in July 2010. There were 5 board
seats up for election but as only 4 nominations were received a vote was not
necessary. All 4 nominations were from existing board members seeking
re-election; Joshua D. Drake. Bdale Garbee, Joerg Jaspert and Martin
Zobel-Helas.

\chapter{Treasury Report}

This report uses a cash-based method of accounting, recording donations when
deposited (not when the check was written or received by us) and recording
expenses when sent or scheduled for payment (not when incurred).

\section{Income Statement}

This covers the Period July 01, 2010 - June 30, 2011

\begin{verbatim}
 Income
   Ordinary Income
        aptosid                             76.68
        Debian                         125,031.07
        DebConf 11                      19,091.00
        Drupal                              89.86
        FreeDesktop.org                 41,896.00
        Gallery                            581.00
        GNU TeXmacs                         60.00
        LibreOffice                     12,000.00
        Madwifi-project.org                 71.00
        OpenOffice.org                   7,361.96
        OpenVas                             51.00
        OpenWrt                            521.86
        Open Voting Foundation             111.00
        OSUNIX                               1.00
        Path64                              11.00
        PostgreSQL                      18,621.86
        Privoxy                            123.00
        SPI General                      6,009.37
        SPI 5%                           1,769.14
        The HeliOS Project               7,940.94
        Tux4Kids                         4,730.45
        Yafaray                          7,136.43

        Total Ordinary Income          253,285.62
                                       ----------

   Interest Income
        Key Bank Money Mkt Savings          20.36
        Chase BusinessClassic Checking       3.76
        Chase Bus Select High Yield Savings 42.00

        Total Interest Income               66.12
                                            -----

   Gross Income                        253,351.74
                                       ----------

 Expenses
  Ordinary Expenses

        aptosid                      9.74
        Debian                 151,634.39
        DebConf 11               5,307.67
        Drupal                   2,739.86
        FreeDesktop.org         31,268.38
        Gallery                     46.72
        GNU TeXmacs                  6.10
        Madwifi-project.org          7.79
        OpenOffice.org             653.20
        OpenVas                      5.29
        OpenWrt                     52.57
        Open Voting Foundation       7.44
        OSUNIX                       0.44
        Path64                       1.69
        PostgreSQL              12,729.30
        Privoxy                     11.85
        SPI                      8,070.54
        The HeliOS Project       8,137.62
        Tux4Kids                 1,470.65
        Yafaray                  2,437.62


        Total Expenses          224,598.86
                                ----------

        Net Income               28,752.88
                                 =========
\end{verbatim}

\section{Balance Sheet}

\begin{verbatim}
As of June 30, 2011

   ASSETS
     Current Assets
        Chase BusinessClassic Checking with Interest      73,433.04
        Chase Business Select High Yield Savings          27,574.42
        Key Business Reward Checking                      72,255.67
        Key Business Signature Money Market Savings       36,012.31
        Ameriprise Cash Mgmt Acct                         13,401.62

     Total Current Assets                                222,677.06


   TOTAL ASSETS                                          222,677.06


   LIABILITIES & EQUITY

     General and current liabilities                           0.00

     Equity
        Reserves held in trust
           aptosid Earmark                       66.94
           Debian Earmark                    55,399.83
           DebConf 11 Earmark                23,783.33
           Freedesktop.org Earmark           12,831.60
           Gallery Earmark                    7,410.80
           GNU TeXmacs Earmark                   63.60
           LibreOffice Earmark               12,000.00
           Madwifi.org Earmark                1,454.09
           OpenOffice.org Earmark            20,965.33
           OpenVAS                               45.71
           OpenWrt                              891.70
           Open Voting Foundation Earmark       153.65
           OSUNIX                                 0.56
           Path64                                 9.31
           Plan 9 Earmark                     6,500.00
           PostgreSQL Earmark                36,882.47
           Privoxy Earmark                      446.50
           The HeliOS Project                   219.20
           Tux4Kids Earmark                   8,788.85
           YafaRay Earmark                    4,897.20

        Total held in trust                              182,810.67

        General reserves                                  39,866.39


     Total Equity                                        222,677.06

   TOTAL LIABILITIES & EQUITY                            222,677.06
\end{verbatim}

(all sums in US dollars)

\chapter{Member Project Reports}

\section{New Associated Projects}

We have continued to see a reasonable level of interest from projects who wish
to become associated with SPI. Over the past year 4 projects have had
successful resolutions proposed for them to be invited to come under the SPI
umbrella as an Associated Project.

\subsection{LibreOffice}

LibreOffice is the free software power-packed personal productivity suite for
Windows, Macintosh and Linux, provided under the LGPLv3/MPL license.

It is made available by The Document Foundation, which is an independent
self-governing meritocratic Foundation.

\subsection{Open64}

The Open64 compiler is an open-source, high performance compiler suite that
targets multiple architectures. It provides a modern framework to implement
advanced optimizations and deliver high performance.

Open64 is the final result of research contributions from a number of compiler
groups around the world. Formerly known as Pro64, Open64 was initially created
by SGI and licensed under the GNU Public License (GPL v2). It was derived from
SGI's MIPSPro compiler.

\subsection{Jenkins}

Jenkins monitors executions of repeated jobs, such as building a software
project or jobs run by cron. Among those things, current Jenkins focuses on the
following two jobs:

\begin{enumerate}
\item Building/testing software projects continuously, just like CruiseControl
or DamageControl. In a nutshell, Jenkins provides an easy-to-use so-called
continuous integration system, making it easier for developers to integrate
changes to the project, and making it easier for users to obtain a fresh build.
The automated, continuous build increases the productivity.

\item Monitoring executions of externally-run jobs, such as cron jobs and
procmail jobs, even those that are run on a remote machine. For example, with
cron, all you receive is regular e-mails that capture the output, and it is up
to you to look at them diligently and notice when it broke. Jenkins keeps those
outputs and makes it easy for you to notice when something is wrong.
\end{enumerate}

\subsection{ankur.org.in}
Ankur.org.in is a group of volunteers who collaborate to promote localization
and internationalization with the specific aim of improving usage of Bengali in
Free and Open Source Software projects. The group is involved in
internationalization and localization especially translations, content
development, development of tools, utilities, widgets and APIs that help
facilitate a wider community of like minded collaborators to participate in a
community building process."

\appendix
\chapter{About SPI}

SPI is a non-profit organization which was founded to help organizations
develop and distribute open hardware and software. We encourage programmers
to use the GNU General Public License or other licenses that allow free
redistribution and use of software, and hardware developers to distribute
documentation that will allow device drivers to be written for their product.

SPI was incorporated as a non-profit organization on June 16, 1997 in the state
of New York. Since then, it has become an umbrella organization for projects
from the community.

In 1999, the Internal Revenue Service (IRS) of the United States government
determined that under section 501 (a) of the Internal Revenue Code SPI
qualifies for 501 (c) (3) (non-profit organization) status under section 509
(a) (1) and 170 (b) (1) (A) (vi). This means that donations made to SPI and its
supported projects should be tax deductible for the American donor.

\end{document}
% Keep this at the bottom, thanks.
% Local Variables:
% TeX-master: "report"
% End:
