\documentclass[letterpaper]{report}
\usepackage{hyperref}

\begin{document}

\title{Software in the Public Interest, Inc.\\
2012 Annual Report}
\date{1st July 2012}

\maketitle

To the membership, board and friends of Software in the Public Interest, Inc:

As mandated by Article 8 of the SPI Bylaws, I respectfully submit this annual
report on the activities of Software in the Public Interest, Inc. and extend my
thanks to all of those who contributed to the mission of SPI in the past year.

  \emph{-- Jonathan McDowell, SPI Secretary}

\newpage

\tableofcontents

\newpage

\chapter{President's Welcome}
\label{sec:president}

  \emph{-- Bdale Garbee, SPI President}

\chapter{Committee Reports}
\section{Membership Committee}

The membership committee was extended to cover the entire board.

\subsection{Statistics}

At the time of writing (July 10th) the current membership status is:

\begin{tabular}{ | c | r | }
\hline
NC Applicants Pending Email Approval	& 94 \\
NC Members				& 485 \\
Contrib Membership Applications		& 11 \\
Contrib Members				& 489 \\
Application Managers			& 9 \\
\hline
\end{tabular}

On 1th July 2011 we had 445 contributing and 436 non-contributing members. On
1st July 2011 there were 481 contributing members and 452 non-contributing
members.

\chapter{Board Report}
\section{Board Members}

Board members as of July 1st, 2011:

\begin{itemize}
\item Bdale Garbee (President)
\item Joerg Jaspert (Vice President)
\item Jonathan McDowell (Secretary)
\item Michael Schultheiss (Treasurer)
\item Joshua D. Drake
\item David Graham
\item Jimmy Kaplowitz
\item Martin Zobel-Helas
\end{itemize}

Board members as of June 30th, 2012:

\begin{itemize}
\item Bdale Garbee (President)
\item Joerg Jaspert (Vice President)
\item Jonathan McDowell (Secretary)
\item Michael Schultheiss (Treasurer)
\item Clint Adams
\item Robert Brockway
\item Joshua D. Drake
\item Jimmy Kaplowitz
\item Martin Zobel-Helas
\end{itemize}

Advisors to the board as of June 30th, 2012:

\begin{itemize}
\item Gregory Pomerantz, legal counsel
\item Software Freedom Law Center (SFLC), legal counsel
\item Stefano Zacchiroli, Debian Project representative
\item Robert Treat, PostgreSQL Project representative
\end{itemize}

\section{Board Changes}

Changes that occurred during the year:

\begin{itemize}
\item The terms for David Graham and Jimmy Kaplowitz expired in July 2011.
Jimmy sought, and obtained, re-election. We thank David for his work on the
board. Clint Adams and Robert Brockway joined the board as part of the same
election.
\item On July 14th, 2011 the board voted to extend the term of the current
officers by a further year. These were:
\begin{itemize}
\item President: Bdale Garbee
\item Vice President: Joerg Jaspert
\item Secretary: Jonathan McDowell
\item Treasurer: Michael Schultheiss
\end{itemize}
\end{itemize}

\section{Elections}

One board membership election was conducted in July 2011. There were 2 board
members up for re-election but due to previous election not filling all
available seats there were 3 seats available. Nominations were received from
Clint Adams, Robert Brockway, Jimmy Kaplowitz and Trevor Walkley. Clint Adams,
Robert Brockway and Jimmy Kaplowitz were elected to the board.

\chapter{Treasury Report}

This report uses a cash-based method of accounting, recording donations when
deposited (not when the check was written or received by us) and recording
expenses when sent or scheduled for payment (not when incurred).

\section{Income Statement}

This covers the Period July 01, 2011 - June 30, 2012

\begin{verbatim}
\end{verbatim}

\section{Balance Sheet}

\begin{verbatim}
As of June 30, 2012
\end{verbatim}

(all sums in US dollars)

\chapter{Member Project Reports}

\section{New Associated Projects}

We have continued to see a reasonable level of interest from projects who wish
to become associated with SPI. Over the past year 6 projects have had
successful resolutions proposed for them to be invited to come under the SPI
umbrella as an Associated Project.


\subsection{Drizzle}

Drizzle is a community-driven open source project that is forked from the
popular MySQL database. The Drizzle team has removed non-essential code,
re-factored the remaining code into a plugin-based architecture and modernized
the code base moving to C++.


\subsection{Arch Linux}

Arch Linux is an independently developed, i686/x86-64 general purpose GNU/Linux
distribution versatile enough to suit any role. Development focuses on
simplicity, minimalism, and code elegance. Arch is installed as a minimal base
system, configured by the user upon which their own ideal environment is
assembled by installing only what is required or desired for their unique
purposes. GUI configuration utilities are not officially provided, and most
system configuration is performed from the shell by editing simple text files.
Arch strives to stay bleeding edge, and typically offers the latest stable
versions of most software.


\subsection{Freedombox}

FreedomBox is a personal server running a free software operating system and
free applications, designed to create and preserve personal privacy by
providing a secure platform upon which federated social networks can be
constructed.

The software for FreedomBox is being assembled by volunteer programmers around
the world who believe in Free Software and Free Society. The FreedomBox
Foundation founded by Eben Moglen coordinates development of a reference
implementation.


\subsection{Fluxbox}

Fluxbox is a windowmanager for X that was based on the Blackbox 0.61.1 code. It
is very light on resources and easy to handle but yet full of features to make
an easy, and extremely fast, desktop experience. It is built using C++ and
licensed under the MIT-License.


\subsection{Haskell.org}

Haskell is an advanced purely-functional programming language. An open-source
product of more than twenty years of cutting-edge research, it allows rapid
development of robust, concise, correct software. With strong support for
integration with other languages, built-in concurrency and parallelism,
debuggers, profilers, rich libraries and an active community, Haskell makes it
easier to produce flexible, maintainable, high-quality software.

Haskell.org is the organisation representing the Open Source Haskell community.


\subsection{FFmpeg}

FFmpeg is a leading multimedia framework, able to decode, encode, transcode,
mux, demux, stream, filter and play pretty much anything that humans and
machines have created. It supports the most obscure ancient formats up to the
cutting edge, no matter if they were designed by some standards committee, the
community, or a corporation. It contains libavcodec, libavutil, libavformat,
libavdevice, libswscale and libswresample which can be used by applications,
along with ffmpeg, ffserver, ffplay and ffprobe programs which can be used
directly by end users for transcoding, streaming and playing.


\appendix
\chapter{About SPI}

SPI is a non-profit organization which was founded to help organizations
develop and distribute open hardware and software. We encourage programmers
to use the GNU General Public License or other licenses that allow free
redistribution and use of software, and hardware developers to distribute
documentation that will allow device drivers to be written for their product.

SPI was incorporated as a non-profit organization on June 16, 1997 in the state
of New York. Since then, it has become an umbrella organization for projects
from the community.

In 1999, the Internal Revenue Service (IRS) of the United States government
determined that under section 501 (a) of the Internal Revenue Code SPI
qualifies for 501 (c) (3) (non- profit organization) status under section 509
(a) (1) and 170 (b) (1) (A) (vi). This means that donations made to SPI and its
supported projects should be tax deductible for the American donor.

\end{document}
% Keep this at the bottom, thanks.
% Local Variables:
% TeX-master: "report"
% End:
