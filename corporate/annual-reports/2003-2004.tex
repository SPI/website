\documentclass[letterpaper]{report}
\usepackage[utf8]{inputenc}
\usepackage{parskip}
\usepackage{hyperref}

\hypersetup{
  colorlinks   = true,
  urlcolor     = blue,
  linkcolor    = blue,
  pdfinfo = {
    Title = {SPI Annual Report 2003-2004},
    Author = {Software in the Public Interest, Inc.},
    Keywords = {SPI, free software, open source, FOSS, annual report, charity, non-profit, 501c3},
  }
}

\begin{document}

\title{Software in the Public Interest, Inc.\\
2004 Annual Report}
\date{Prepared as of July 1, 2004 on August 13, 2004}

\maketitle
To the membership, Board, and friends of Software in the Public Interest,
Inc.:

As mandated by Article 8 of the SPI Bylaws, I respectfully submit
this annual report on the activities of Software in the Public Interest,
Inc., and extend my thanks to all those that contributed to the mission
of SPI in the past year.

\emph{-- John Goerzen, SPI President}

\newpage

\tableofcontents

\newpage

\chapter{Introduction}

At the July 1, 2004 \href{https://www.spi-inc.org/}{Software in the Public Interest, Inc.}
(SPI) annual meeting, the President informed the Board that an annual
report was not yet complete. The President was asked to collect submissions
and finish the annual report. Then-President Ean Schuessler's term
expired before the report could be completed. Incoming President John
Goerzen completed the report and presented it.

SPI's 2003 annual meeting was held July 8, 2003. The 2004 meeting
was held July 1, 2004.

\emph{Submitted by John Goerzen, Report Editor}


\section{President's Letter}
\label{sec:president}

Fellow Hackers and Friends,

SPI's seven short years have been a wild ride. We've seen Free Software
grow from a philosophy into a movement that is becoming a way of life.
We've truly come of age. Every IT organization in the world has a
Free Software strategy or is under working to adopt one. Linux provided
Microsoft the adversary no one thought possible. Fortunes were won
and lost. Powerful new friends appeared. Legal pitfalls, bad legislation
and surprising traitors have all so far failed to stop the phenomena
we all love. SPI survived these stormy seas and is still working to
bring commercial quality support services to Free Software projects
everywhere.

We have had our own internal challenges as well. The runaway growth
of the movement has tested the will and even the physical ability
of our all volunteer staff to the limit and beyond. We've made mistakes
and even had a few disasters. None of these problems have proven fatal
thanks to the dedication of our membership, officers, supporters and
pro-bono volunteers. The journey has been a learning process.

We've had some real successes along the way as well. Debian, the shining
star of SPI projects, has continued its rocket powered growth. In
2004, Netcraft Ltd. announced that Debian is the fastest growing web
server operating system in the world. The Brazilian government announced
comprehensive plans to implement Debian nationally. New maintainers
and the number of available packages continues to grow on a daily
basis.

This is truly exciting and challenging time to be involved with SPI.
The potential of Free Software is no longer a dream. It is a reality.
Every day more and more people depend on Free Software to communicate
with their friends and family, to run their businesses and to satisfy
their creative urge. With that success comes a great new responsibility.
Every new Free Software project must show professional competency
and commitment or face failure. Today, more than ever, there is a
need for SPI and the services it provides.

I'm truly proud to have played a part and thank each of you for the
opportunity. Let's make the next seven years just as unbelievably
amazing.

Happy Hacking,

\emph{Submitted by Ean Schuessler, SPI President}


\chapter{Committee Reports}

SPI has the following chartered committees:

\begin{itemize}
\item Project Committee, chartered by \href{https://www.spi-inc.org/corporate/resolutions/1999/1999-09-21.iwj}{1999-09-21.iwj}
\item Open Source Committee, chartered by \href{https://www.spi-inc.org/corporate/resolutions/2001/2001-04-26.nl}{2001-04-26.nl}
\item Membership Committee, chartered by \href{https://www.spi-inc.org/corporate/resolutions/2001/2001-08-06.nl}{2001-08-06.nl}.
This is a permanent committee per Article 10 of the SPI Bylaws.
\item Trademark Committee, chartered by \href{https://www.spi-inc.org/corporate/resolutions/2003/2003-06-03.bmh.1}{2003-06-03.bmh.1}
\item Bylaws Committee, chartered by \href{https://www.spi-inc.org/corporate/resolutions/2003/2003-01-06.wta.2}{2003-01-06.wta.2}
, extended by \href{https://www.spi-inc.org/corporate/resolutions/2003/2003-06-27.jpg.1}{2003-06-27.jpg.1}
and made a standing committee by \href{http://lists.spi-inc.org/pipermail/spi-announce/2004/000067.html}{2004-01-05.dbg.1}
\end{itemize}

Additionally, SPI has one unchartered permanent committee mandated
by Article 10 of the SPI Bylaws: the administration committee.

\emph{Committee summary submitted by John Goerzen}

\vspace{1em}

The following reports were submitted by or about SPI committees:

\section{Bylaws Committee}

The Bylaws committee was established with resolution \href{https://www.spi-inc.org/corporate/resolutions/2003/2003-01-06.wta.2}{2003-01-06.wta.2}
with a time-limited charter. The committee's purpose is to review
SPI's bylaws and advance specific suggestions for improvement. The
charter was extended in 2003, and on January 5, 2004, the Board adopted
resolution \href{http://lists.spi-inc.org/pipermail/spi-announce/2004/000067.html}{2004-01-05.dbg.1},
which removed the expiration provision.

The committee issued its first bylaws amendment \href{http://gopher.quux.org:70/devel/bylaws/20030701.html}{recommendations}
on July 1, 2003 according to its original charter. A vote was not
held immediately because the membership committee was in the process
of ensuring the integrity of SPI's membership roster. In January,
2004, the Board adopted resolution \href{http://lists.spi-inc.org/pipermail/spi-announce/2004/000066.html}{2004-01-06.wta.4},
calling for a vote among the membership once the membership committee's
work was complete. That vote \href{http://lists.spi-inc.org/pipermail/spi-announce/2004/000071.html}{began}
on March 2, 2004.

While the vote was in progress, the Board received advice both from
SPI's legal council and others that certain provisions in the new
proposal may need adjustment for full legal compliance. On March 2,
2004, the Board \href{http://lists.spi-inc.org/pipermail/spi-announce/2004/000073.html}{suspended}
the vote until these issues could be fully considered. As of July
1, that analysis has not yet been completed.

The members of the Bylaws Committee \href{http://lists.spi-inc.org/pipermail/spi-bylaws/2003/000003.html}{are}:
John Goerzen (chair), JP ``Taral'' Sugarbroad (secretary), Manoj
Srivastava, David Graham, and Jimmy Kaplowitz.

\emph{Submitted by John Goerzen, Chair of the Bylaws Committee}


\section{Membership Committee}

The Membership Committee is one of the two permanent committees recognized
by Article 10 of the SPI Bylaws. The Membership Committee was originally
chartered by \href{https://www.spi-inc.org/corporate/resolutions/2001/2001-08-06.nl}{2001-08-06.nl}.

SPI presently has 551 members, 290 (53\%) of which are contributing
members. Since July 1, 2003, SPI has had a net gain of 148 total members,
70 (47\%) of which are contributing members.

During the year, concern arose that SPI membership rolls contained
inactive members. Large numbers of inactive members could artificially
raise quorum requirements for membership votes, potentially making
it difficult to conduct an actual vote. The Membership Committee sought
to validate active members. However, very few members were determined
to actually be inactive, and there was almost no impact on SPI membership
rolls.

The members of the Membership Committee \href{https://www.spi-inc.org/meetings/minutes/2003/2003-03-11/}{are}:
Martin Michlmayr (chair), Craig Small, and Peter Palfrader.

\emph{Submitted by John Goerzen based on information gathered by Peter
Palfrader}


\chapter{Board Report}


\section{Board Members}

Board members as of July 8, 2003:

\begin{itemize}
\item Ean Schuessler (President)
\item John Goerzen (Vice President)
\item Wichert Akkerman (Secretary)
\item Branden Robinson (Treasurer)
\item Ian Jackson
\item Martin ``Joey'' Schulze
\item M. Drew Streib
\item Manoj Srivastava
\item Bruce Perens
\item Benjamin Mako Hill
\end{itemize}

Board members as of July 1, 2004:

\begin{itemize}
\item Ean Schuessler (President) (term expires July 1)
\item John Goerzen (Vice President)
\item Wichert Akkerman (Secretary) (term expires July 1)
\item Jimmy Kaplowitz (Treasurer)
\item Branden Robinson
\item Ian Jackson
\item Martin ``Joey'' Schulze
\item Bruce Perens
\item Benjamin Mako Hill
\item David Graham
\end{itemize}

\section{Board Changes}

Changes that occurred during the year:

\begin{itemize}
\item Manoj Srivastava resigned (source: \href{https://www.spi-inc.org/corporate/resolutions/2003/2003-10-14.iwj.6}{2003-10-14.iwj.6}).
\item David Graham was \href{http://lists.spi-inc.org/pipermail/spi-announce/2003/000065.html}{elected}
to the board on November 29, 2003 in an election initiated by \href{https://www.spi-inc.org/corporate/resolutions/2003/2003-10-14.iwj.6}{2003-10-14.iwj.6}.
\item Jimmy Kaplowitz was \href{http://lists.spi-inc.org/pipermail/spi-announce/2004/000070.html}{appointed}
as new treasurer on February 20, 2004, having run unopposed in an
election for that position.
\item M. Drew Streib was removed by the Board on July 1, 2004 (resolution
\href{http://lists.spi-inc.org/pipermail/spi-announce/2004/000083.html}{2004-07-01.dbg.2})
due to the board meeting attendance policy.
\end{itemize}

\section{Elections}

One membership election for board seats was conducted in the past
year. It began in October 2003 as a result of \href{http://lists.spi-inc.org/pipermail/spi-announce/2003/000057.html}{2003-10-14.iwj.6}.
Three seats were up for selection and four candidates ran for office.
Two of them, Martin ``Joey'' Schulze and Ian Jackson were seeking
re-election. The other two, David Graham and Jimmy Kaplowitz, were
seeking new terms on the Board. 52 people voted in this election,
selecting Martin ``Joey'' Schulze, Ian Jackson, and David Graham
for the three available seats.

In January 2004, the Board passed \href{http://lists.spi-inc.org/pipermail/spi-announce/2004/000069.html}{2004-01-06.jrk.1.br.1},
establishing an election for the Treasurer position. However, only
one candidate volunteered for the position. Therefore, that candidate,
Jimmy Kaplowitz, was selected as treasurer without the need to run
a full election.

\emph{This report submitted by John Goerzen, Chair of the Board}


\chapter{Treasury Report}


\section{Accounting Issues}

Volunteer accounting services have been a serious problem for SPI
virtually since its inception. In 2004, SPI initiated a comprehensive
audit of its past financial records. Brainfood, Inc., a Texas corporation,
provided \emph{pro-bono} clerical assistance in processing the backlog
of mail that had overwhelmed the volunteer SPI Treasurers.

\emph{Submitted by Ean Schuessler}


\subsection{Audit Results Summary}

The audit revealed more than \$18,000 of uncashed donation checks.
Other investigations by the SPI Treasurer revealed SPI was the victim
of some fraud due to publishing the SPI bank account number for online
donations.

\emph{Submitted by Ean Schuessler}


\subsection{Corrective Actions}

The Board and membership engaged in extensive discussions about the
causes of and best methods to address the problems encountered. The
Board authorized apology letters to be sent to all contributors that
could be reached, along with their original checks. Ean reported to
the Board on May 18 that the letters would go out that week. Checks
that had not yet expired were to be deposited as soon as possible.

On \href{http://lists.spi-inc.org/pipermail/spi-announce/2004/000074.html}{May 4},
the Board affirmed, via resolution 2004-05-03.bp.2, the ability of
the President and Treasurer to work together to take the necessary
steps to solve the problem. At the same meeting, the Board approved
resolution 2004-05-04.dbg.2, giving the Treasurer a limited budget
and discretion to take necessary steps to solve problems without further
Board approval, with an expiration in six months.

\emph{Submitted by John Goerzen}


\subsection{Preventing Re-Occurrence}

The Board, Treasurer, officers, and membership continue to work on
ways to make our organization resilient in the face of such problems
in the future. Greater transparency has been asked of our treasurer,
and various proposals for professional or voluntary assistance are
under consideration. Some have expressed the desire to wait with major
structural changes until we have passed the immediate problem and
can more clearly analyze the specific failures.

The Board authorized the new treasurer to obtain a Post Office box
dedicated solely to SPI matters to provide a stable correspondence
address. The Board also authorized expedited posting of draft meeting
minutes to help keep the membership better informed of SPI news.

SPI welcomes improvement suggestions and volunteer contributions.

\emph{Submitted by John Goerzen}


\section{Account Balances}

SPI maintains these three accounts:

\begin{tabular}{|c|c|c|}
\hline
Description & Balance & Last Report Date\tabularnewline
\hline
\hline
American Express Bank, FSB Checking, Debian account & \$568.64 & June 30, 2004\tabularnewline
\hline
American Express Bank, FSB Checking, SPI account & \$238.27 & June 30, 2004\tabularnewline
\hline
American Express Financial Advisors AXP Cash Mgmt. Fund & \$39,504.43 & July 13, 2004\tabularnewline
\hline
\end{tabular}

\emph{Submitted by John Goerzen based on earlier reports from Branden Robinson}


\section{Project Allocations}

SPI presently holds funds exceeding \$1 for only two groups: Debian
and itself, though through the year it has also handled deposits and
withdrawals for wxWidgets. The accounting issues documented elsewhere
in this report mean that exact figures are not presently available.
However, the SPI Treasurer plans to begin a comprehensive audit shortly
to ensure exact figures are available.


\subsection{Checking Accounts}

SPI holds two checking accounts: one for Debian funds and one for
SPI funds. It is thought that these amounts are roughly accurate,
though the Treasurer stressed they should be audited to confirm this.


\subsection{Cash Management Fund}

Prior to January 2002, records for specific allocations in the AXP
Cash Management Fund were not kept. In April 2003, the Board passed
\href{https://www.spi-inc.org/corporate/resolutions/2003/2003-04-01.wta.1}{2003-04-01.wta.1},
which allocated 95\% of the funds in that account as of January 1,
2002 to Debian and 5\% to SPI. Actual numbers are to be used after
that date.

While records are believed to exist for the transactions that have
occurred since 2002, running balances have not been kept up to date,
so audits are needed to determine exact amounts. Present best guesses
indicate that 95\% or slightly less of that fund still is earmarked
for Debian.


\subsection{Totals}

Assuming a 95\%/5\% split for the AXP Cash Management Fund, SPI has
approximately \$2,200 and Debian approximately \$38,000. Again, the
Treasurer has stressed that these numbers are un-audited and represent
guesses only.

\emph{This report submitted by John Goerzen based on available records
and interviews with current Treasurer Jimmy Kaplowitz and former Treasurer
Branden Robinson}


\section{Accepting Donations}

On September 9, 2003, the board passed \href{https://www.spi-inc.org/corporate/resolutions/2003/2003-09-09.wta.1}{2003-09-09.wta.1}
authorizing the use of \href{http://www.networkforgood.org/}{Network For Good}
to accept credit card donations online. On November 4, 2003, SPI \href{https://www.spi-inc.org/news/2003/20031104}{announced}
that online credit card donations were now available. SPI's \href{https://www.spi-inc.org/donations}{donations}
page contains information about online and offline methods of making
donations.

\emph{Submitted by John Goerzen}


\section{SPI-Related Hardware Assets Report}

SPI maintains the following four machines:

\begin{tabular}{|c|c|c|}
\hline
Machine & Hardware Donor & Hosting/Bandwidth Donor\tabularnewline
\hline
\hline
purcel & VA Linux Systems & Brainfood, Inc.\tabularnewline
\hline
frida & Oregon State Univ. Open Source Lab & Oregon State Univ. Open Source Lab\tabularnewline
\hline
styx & Oregon State Univ. Open Source Lab & Oregon State Univ. Open Source Lab\tabularnewline
\hline
chic & bash.sh, ltd. & bash.sh, ltd.\tabularnewline
\hline
\end{tabular}

\emph{Submitted by John Goerzen based on SPI's \href{https://www.spi-inc.org/credits}{credits page}}


\section{Debian-Related Hardware Assets Report}


\subsection{Hardware Donations Overview}

The Debian project uses a large amount of dedicated hardware. The
vast majority of this hardware is not owned by Software in the Public
Interest but is rather loaned to the project by developers or corporations
interested in helping the project or given to individual developers
for use on Debian.

Because donations of hardware can be deducted from income on tax purposes,
a number of individuals and corporations have expressed an interest
in donating hardware -- especially older hardware -- to the Debian
project through Software in the Public Interest.

As a result, SPI holds a small amount of hardware as assets.

To track hardware donations, Bdale Garbee in his tenure as Debian
Project Leader appointed Benjamin Mako Hill as a hardware donations
delegate in 2002 to track hardware donations for the Debian project.
He was later joined by Rob Bradford in the position. Both continue
to field requests related to hardware donations emailed to <hardware-donations@debian.org>.

In spite of several requests to debian-admin, the group has been unable
to get a good picture of the ownership status of older Debian machines.


\subsection{Hardware Donation Process}

The normal process for hardware donations to Debian follows the following
procedure:

\begin{itemize}
\item Hardware is offered to hardware-donations@debian.org;
\item A delegate will determine whether or not the hardware is usable for
Debian;
\item If someone in the project is waiting for the type of hardware being
offered, the donor and the potential recipient will be connected to
coordinate shipping;
\item Otherwise, the hardware will be offered on a public list and responses
from developers will be solicited. The delegate will help select good
potential recipients from the replies;
\item Once a recipient has been found for a given piece of hardware, the
hardware will be shipped or delivered, almost always at the expense
of the donor, to the recipient;
\item If the donor wants or needs a receipt, they will be provided one signed
by the SPI treasurer and a Debian hardware donations delegate;
\end{itemize}

Only in the last case will Software in the Public Interest be involved
in the process. In these cases, the donor will be asked to approximate
the value of the hardware. It is up the donor to value this correctly
since they will be deducting this amount from their taxes.

After this, the value of the hardware is be devalued by 1/3 of its
value on its anniversary each year for 3 years, at which point it
will be considered to have no value.


\subsection{Recent Hardware Donations}

SPI has only been asked to provide for the following two donations
in the last year:

\begin{tabular}{|c|c|c|c|}
\hline
Date & Name & Description & Value\tabularnewline
\hline
\hline
2003-03-03 & John Gabriel & Sun Blade & \$800\tabularnewline
\hline
2003-08-02 & Daniel Silverstone & MIPS Board & \$1000\tabularnewline
\hline
\end{tabular}

\emph{This report submitted by Benjamin Mako Hill}


\chapter{Member Project Reports}

During the past year, wxWidgets (formerly wxWindows) was invited to
become an SPI member project. The invitation was accepted.

The following report was received from an SPI member project:


\section{Debian Project}

The Debian Project is an association of individuals who have made
common cause to create a free operating system. This operating system
that we have created is called Debian GNU/Linux, or simply Debian
for short. Debian is by far the most significant SPI project and represents
the majority of SPI's membership and financial transactions.

\emph{Submitted by Ean Schuessler}

\appendix

\chapter{About SPI}

SPI is a non-profit organization which was founded to help organizations
develop and distribute open hardware and software. We encourage programmers
to use the GNU General Public License or other licenses that allow
free redistribution and use of software, and hardware developers to
distribute documentation that will allow device drivers to be written
for their product.

SPI was incorporated as a non-profit organization on June 16, 1997
in the state of New York. Since then, it has become an umbrella organization
for projects from the community.

In 1999, the Internal Revenue Service (IRS) of the United States government
determined that under section 501 (a) of the Internal Revenue Code
SPI qualifies for 501 (c) (3) (non-profit organization) status under
section 509 (a) (1) and 170 (b) (1) (A) (vi). This means that donations
made to SPI and its supported projects should be tax deductible for
the American donor.

\emph{Submitted by Ean Schuessler}

\end{document}

