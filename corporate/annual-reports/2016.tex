\documentclass[letterpaper]{report}
\usepackage[utf8]{inputenc}
\usepackage{parskip}
\usepackage{hyperref}

\hypersetup{
  colorlinks   = true,
  urlcolor     = blue,
  linkcolor    = blue,
  pdfinfo = {
    Title = {SPI Annual Report 2016},
    Author = {Software in the Public Interest, Inc.},
    Keywords = {SPI, free software, open source, FOSS, annual report, charity, non-profit, 501c3},
  }
}

\begin{document}

\title{Software in the Public Interest, Inc.\\
2016 Annual Report}
\date{July XXX, 2017}

\maketitle

To the membership, board and friends of Software in the Public Interest, Inc:

As mandated by Article 8 of the SPI Bylaws, I respectfully submit this annual
report on the activities of Software in the Public Interest, Inc. and extend my
thanks to all of those who contributed to the mission of SPI in the past year.

  \emph{-- Martin Michlmayr, SPI President}

\newpage

\tableofcontents

\newpage

\chapter{President's Welcome}
\label{sec:president}

  \emph{-- Martin Michlmayr, SPI President}

\chapter{Committee Reports}
\section{Membership Committee}

\subsection{Statistics}

On January 1, 2016 we had 517 contributing and 520 non-contributing members.
On December 31, 2016 there were XXX contributing members and XXX non-contributing
members.


\chapter{Board Report}
\section{Board Members}

Board members as of January 1, 2016:

\begin{itemize}
\item Bdale Garbee (President)
\item Joerg Jaspert (Vice President)
\item Martin Michlmayr (Secretary)
\item Michael Schultheiss (Treasurer)
\item Robert Brockway
\item Joshua D. Drake
\item Dimitri John Ledkov
\item Gregers Petersen
\item Martin Zobel-Helas
\end{itemize}

Board members as of December 31, 2016:

\begin{itemize}
\item Martin Michlmayr (President)
\item Joerg Jaspert (Vice President)
\item Valerie Young (Secretary)
\item Michael Schultheiss (Treasurer)
\item Luca Filipozzi
\item Jimmy Kaplowitz
\item Dimitri John Ledkov
\item Andrew Tridgell
\item Martin Zobel-Helas
\end{itemize}

Advisors to the board as of December 31, 2016:

\begin{itemize}
\item Software Freedom Law Center (SFLC), legal counsel
\item Neil McGovern, Debian Project representative
\item Robert Treat, PostgreSQL Project representative
\end{itemize}

\section{Board Changes}

Changes that occurred during the year:

\begin{itemize}

\item Robert Brockway resigned from the board at the end of May 2016 due
to lack of time.  We'd like to thank him for his many contributions over
the years!

\item Gregers Petersen resigned from the board in June 2016 due to lack
of time.  We'd like to thank Gregers for his contributions!

\item The terms for Joshua D. Drake, Bdale Garbee, Joerg Jaspert and
Martin Zobel-Helas expired in July 2016.  Joerg and Martin sought, and
obtained, re-election.  We'd like to thank Joshua D. Drake and Bdale
Garbee for their work on the board.  Luca Filipozzi, Jimmy Kaplowitz,
Andrew Tridgell and Valerie Young joined the board as part of the same
election.

\item On August 11, 2016 the board voted to appoint the following
officers:

\begin{itemize}
\item President: Martin Michlmayr
\item Vice President: Joerg Jaspert
\item Secretary: Valerie Young
\item Treasurer: Michael Schultheiss
\end{itemize}

\end{itemize}

\section{Elections}

A board membership election was conducted in July 2016.  There were 6
board seats up for election.  Nominations were received from Philip
Balister, R. Tyler Croy, Joshua D. Drake, Peter Eisentraut, Luca
Filipozzi, Stephen Frost, Joerg Jaspert, Jimmy Kaplowitz, Tim Potter,
Craig Small, Andrew Tridgell, Valerie Young, and Martin Zobel-Helas.

Luca Filipozzi, Joerg Jaspert, Jimmy Kaplowitz, Andrew Tridgell, Valerie
Young and Martin Zobel-Helas were elected to the board.

\chapter{Treasury Report}

This report uses a cash-based method of accounting, recording donations when
deposited (not when the check was written or received by us) and recording
expenses when sent or scheduled for payment (not when incurred).

\section{Income Statement}

This covers the Period January 1, 2016 -- December 31, 2016

\begin{verbatim}
TBD
\end{verbatim}

\section{Balance Sheet}

\begin{verbatim}
Balance Sheet as of December 31, 2016

TBD
\end{verbatim}

\chapter{Member Project Reports}

\section{New Associated Projects}

We have continued to see a reasonable level of interest from projects
who wish to become associated with SPI. Over the past year, 8 projects
joined the SPI umbrella as an Associated Project.

\subsection{ArduPilot}

ArduPilot is a cross-platform free software autopilot project for all
types of small robotic vehicles. With a very active developer and user
community ArduPilot provides sophisticated navigation and control for
all types of flying vehicles, boats and ground vehicles.

\subsection{Glucosio}

Glucosio is an open source project dedicated to bringing open source
apps to smartphone, desktop and web in order to help people with
diabetes improve their health outcomes by better self-management of
their disease. At the same time, Glucosio offers the opportunity for
opt-in to crowdsourcing of anonymized health trends and demographics to
support diabetes research. Apps are dual licensed under the GPLv3/MPL
2.0 license.

\subsection{NTPsec}

The NTPsec project is a more secure, hardened, and improved
implementation of Network Time Protocol derived from NTP Classic, Dave
Mills's original. We employ best practices and state-of-the art
technology in code auditing, verification, and testing to deliver code
that can be used with confidence in deployments with the most stringent
security, availability, and assurance requirements.

\subsection{OpenZFS}

OpenZFS is an umbrella project which brings together individuals and
companies that use and improve the ZFS file system, and encourages the
widespread use of ZFS and its development in a true open-source manner.

ZFS is storage software which combines the functionality of traditional
filesystems, volume manager, and more. ZFS include protection against
data corruption, support for high storage capacities, efficient data
compression, snapshots and copy-on-write clones, continuous integrity
checking and automatic repair, remote replication with ZFS send and
receive, and RAID-Z.

\subsection{Torch}

Torch is a scientific computing framework with wide support for machine
learning algorithms that puts GPUs first. It is easy to use and
efficient, thanks to an simple and fast scripting language, Lua, and an
underlying C/CUDA implementation.

\subsection{X.Org}

The X.Org community creates a free and open accelerated graphics stack,
including major components such as the DRM kernel graphics subsystem,
Mesa 3D graphics library, Wayland compositor and the X.Org Window
System.

\subsection{Performance Co-Pilot}

Performance Co-Pilot (PCP) provides a framework and services to support
system-level performance monitoring and management. It presents a
unifying abstraction for all of the performance data in a system, and
many tools for interrogating, retrieving and processing that data.

PCP is a feature-rich, mature, extensible, cross-platform toolkit
supporting both live and retrospective analysis. The distributed PCP
architecture makes it especially useful for those seeking centralized
monitoring of distributed processing.

\subsection{Open MPI}

The Open MPI Project is an open source Message Passing Interface
implementation that is developed and maintained by a consortium of
academic, research, and industry partners. Open MPI is therefore able to
combine the expertise, technologies, and resources from all across the
High Performance Computing community in order to build the best MPI
library available. Open MPI's plugin-based architecture offers
advantages for system and software vendors, application developers and
computer science researchers.


\section{Updates from Associated Projects}


\appendix
\chapter{About SPI}

SPI is a non-profit organization which was founded to help organizations
develop and distribute open hardware and software. We encourage programmers
to use the GNU General Public License or other licenses that allow free
redistribution and use of software, and hardware developers to distribute
documentation that will allow device drivers to be written for their product.

SPI was incorporated as a non-profit organization on June 16, 1997 in the state
of New York. Since then, it has become an umbrella organization for projects
from the community.

In 1999, the Internal Revenue Service (IRS) of the United States government
determined that under section 501 (a) of the Internal Revenue Code SPI
qualifies for 501 (c) (3) (non-profit organization) status under section 509
(a) (1) and 170 (b) (1) (A) (vi). This means that donations made to SPI and its
supported projects should be tax deductible for the American donor.

\end{document}
% Keep this at the bottom, thanks.
% Local Variables:
% TeX-master: "report"
% End:
