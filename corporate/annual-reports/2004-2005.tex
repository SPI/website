\documentclass[letterpaper]{report}
\usepackage[utf8]{inputenc}
\usepackage{parskip}
\usepackage{hyperref}

\hypersetup{
  colorlinks   = true,
  urlcolor     = blue,
  linkcolor    = blue,
  pdfinfo = {
    Title = {SPI Annual Report 2004-2005},
    Author = {Software in the Public Interest, Inc.},
    Keywords = {SPI, free software, open source, FOSS, annual report, charity, non-profit, 501c3},
  }
}

\begin{document}

\title{Software in the Public Interest, Inc.\\
2005 Annual Report}
\date{Prepared for\\
July 1, 2005 Annual Meeting}

\maketitle

To the membership, Board, and friends of Software in the Public Interest,
Inc.:

As mandated by Article 8 of the SPI Bylaws, I respectfully submit this
annual report on the activities of Software in the Public Interest, Inc.,
and extend my thanks to all those that contributed to the mission of SPI in
the past year.

\emph{-- John Goerzen, SPI President}

\newpage

\tableofcontents

\newpage

\chapter{President's Welcome}
\label{sec:president}

``What does SPI do?''

That's a question I hear frequently. I have a standard answer: Software in
the Public Interest, Inc. (SPI) holds assets on behalf of various Free
Software projects. But I think that is not the question, or the answer,
that is most interesting.

``Why does SPI do this?''

This is a much more interesting question. SPI, at its very core, is about
freedom. There are many organizations out there that promote freedom. SPI
promotes freedom through software. This is not an abstract academic notion.
By promoting freedom through software, SPI is on the leading edge of an
ongoing battle. SPI believes in freedom to share ideas, freedom to modify
something one owns, and most importantly, freedom to develop new
technologies by standing on the shoulders of giants. Many forces seek to
restrict these freedoms.  Proprietary software licensing agreements often
prohibit users not just from modifying the software they have purchased,
but from even attempting to learn how it works.

We believe that the world would be better served by preserving the freedoms
of an earlier era in this digital age. We believe that Free Software holds
great benefits in terms of quality, education, and price.

I believe SPI, and the greater Free Software community, are powerful forces
for good. I would like to personally invite you to become involved with SPI
if you aren't already. Almost all of our activities are handled and managed
by volunteers, and we are always seeking to expand participation and
interest in SPI.

Thank you all for your hard work in this cause.

\emph{Submitted by John Goerzen, SPI President}

\chapter{Committee Reports}

SPI has the following chartered committees:

\begin{itemize}

\item Open Source Committee, chartered by 2001-04-26.nl
\item Membership Committee, chartered by 2001-08-06.nl. This is a permanent committee per Article 10 of
the SPI Bylaws.
\item Trademark Committee, chartered by 2003-06-03.bmh.1
\item Bylaws Committee, chartered by 2003-01-06.wta.2 , extended by 2003-06-27.jpg.1 and made a standing
committee by 2004-01-05.dbg.1

\end{itemize}

Additionally, SPI has one unchartered permanent committee mandated by
Article 10 of the SPI Bylaws: the administration committee.

On May 17, the Board abolished the Projects committee, which had been
inactive for some time.

\emph{Committee summary submitted by John Goerzen}

\vspace{1em}

The following reports were submitted by or about SPI committees:

\section{Membership Committee}

Craig Small originally built the web site based on his scripts for Debian's
NM process. Later, Martin Michlmayr and Peter Palfrader took over; little
modifications to the system itself were made but Peter added support for
handling votes. In April 2005, they indicated their interest to step down
and in May Graham Wilson, Luk Claes and Michael Schultheiss became the new
Membership Committee.

\subsection{Statistics}

Here are some statistics from the membership committee:

\begin{tabular}{|c|r|}

\hline
NC Applicants Pending E-mail Approval & 31 \tabularnewline
\hline
Non-Contributing Members & 282 \tabularnewline
\hline
Contributing Membership Applications & 8 \tabularnewline
\hline
Contributing Members & 313 \tabularnewline
\hline
Application Managers & 6 \tabularnewline
\hline

\end{tabular}


\subsection{Open Questions}

In the past, Martin Michlmayr (head of the Membership Committee at that
time) became an adviser to the SPI board. It is not clear whether the new
head should become an adviser too since this is not the case with other
groups within SPI.  This is something the SPI Board has to decide.

\subsection{To-Do List}

\begin{itemize}
\item Move the PHP code from CVS to SVN
\item Check all the scripts into SVN as well, and re-structure the layout (move the *.php scripts in a subdirectory)
\item Create an archive for the Membership Committee alias.
\item membership@spi-inc.org should probably just be an alias to members@members.spi-inc.org so the SPI admin doesn't have to be bothered when changes are needed.
\end{itemize}

\emph{Submitted by Martin Michlmayr, outgoing chair of the Membership Committee}


\chapter{Board Report}

\section{Board Members}

Board members as of July 1, 2004:

\begin{itemize}
\item Ean Schuessler (President) (term expires July 1, 2004)
\item John Goerzen (Vice President)
\item Wichert Akkerman (Secretary) (term expires July 1, 2004)
\item Jimmy Kaplowitz (Treasurer)
\item Branden Robinson
\item Ian Jackson
\item Martin ``Joey'' Schulze
\item Bruce Perens
\item Benjamin Mako Hill
\item David Graham
\end{itemize}

Board members as of June 30, 2005:

\begin{itemize}
\item John Goerzen (President)
\item Benjamin Mako Hill (Vice President)
\item David Graham (Secretary)
\item Jimmy Kaplowitz (Treasurer)
\item Bdale Garbee
\item Branden Robinson
\item Ian Jackson
\item Bruce Perens
\item Martin ``Joey'' Schulze
\end{itemize}

Advisors to the board as of June 30, 2005:

\begin{itemize}
\item Gregory Pomerantz, legal counsel
\item Jeff Waugh, GNOME Foundation Board representative
\end{itemize}

\section{Board Changes}

Changes that occurred during the year:

\begin{itemize}

\item The terms for Wichert Akkerman and Ean Schuessler expired on July 1, 2004. Neither sought reelection.
\item Bdale Garbee and Branden Robinson were elected to the board on July 29, 2004.
\item On August 10, 2004, the Board selected the following officers for the coming year:

	\begin{itemize}

	\item President: John Goerzen
	\item Vice President: Benjamin Mako Hill
	\item Secretary: David Graham
	\item Treasurer: Jimmy Kaplowitz

	\end{itemize}

\end{itemize}

\section{Elections}

One membership election for board seats was conducted in the past year. It
ended in August 2004, selecting Bdale Garbee and Branden Robinson for seats
on the board.

In August 2004, the Board approved 2004-08-10.dbg.2, with provided a
regular framework for Board member elections.

\emph{This report submitted by John Goerzen, Chair of the Board}


\chapter{Treasurer's Report}

\section{Accounting \& Taxes}

SPI's 2004 annual report reflected a desire to improve the timeliness of
processing financial documents, as well as the implementation of procedures
to make sure we provide the best possible service in the future.  Over the
course of the past year, much progress has been made on this task, and it
has been the task receiving the single greatest attention of the Board over
the past year.  It is clear that SPI still needs to improve.

The Board and SPI Treasurer have taken the following actions over the past
year to improve the situation:

\begin{itemize}
\item Switched to a more responsive bank, First Internet Bank of Indiana, in December
\item Established a permanent legal address with a registered agent in the State of New York
\item On October 28, established a regular budget for the Treasurer, providing for any expenses necessary
to carry out work related to the organizations finances without obtaining Board approval each time
\item The SPI Treasurer began issuing regular monthly income, expense, and balance reports
\item On June 21, authorized a contract with Mark's Bookkeeping Service in Manhattan to help get our
existing books in proper order and assist with day-to-day financial management and future reports
\item Also on June 21, authorized retaining an CPA for advice on tax filing requirements in the State of New York
\end{itemize}

\emph{Submitted by John Goerzen}

\section{Account Balances}

2005 balances are current as of May 30, 2005.

\begin{tabular}{|c|r|r|}

\hline
Description & Balance in 2004 AR & 2005 Balance \tabularnewline
\hline
\hline
AmEx Bank, FSB Checking, Debian account & \$568.64 & n/a \tabularnewline
\hline
AmEx Bank, FSB Checking, SPI account & \$238.27 & n/a \tabularnewline
\hline
AmEx Financial Advisors AXP Cash Mgmt. Fund & \$39,504.43 & \$39,882.62 \tabularnewline
\hline
First Internet Bank of Indiana, Checking & n/a & \$12,901.34 \tabularnewline
\hline
Totals & \$40,311.34 & \$52,783.96 \tabularnewline
\hline

\end{tabular}

Unfortunately, a detailed financial report was not available for this
year's annual report.

\emph{Submitted by John Goerzen based on the 2004 Annual Report and the
Treasurer's latest report}

\chapter{Member Project Reports}

During the past year, Drupal and GNUstep joined SPI as member projects.

The following report was received from an SPI member project:

\section{Debian Project}

The Debian Project is an association of individuals who have made common
cause to create a free operating system. This operating system that we have
created is called Debian GNU/Linux, or simply Debian for short. Debian is
by far the most significant SPI project and represents the majority of
SPI's membership and financial transactions.

\emph{Submitted by Ean Schuessler}

\appendix

\chapter{About SPI}

SPI is a non-profit organization which was founded to help organizations
develop and distribute open hardware and software. We encourage programmers
to use the GNU General Public License or other licenses that allow free
redistribution and use of software, and hardware developers to distribute
documentation that will allow device drivers to be written for their
product.

SPI was incorporated as a non-profit organization on June 16, 1997 in the
state of New York. Since then, it has become an umbrella organization for
projects from the community.

In 1999, the Internal Revenue Service (IRS) of the United States government
determined that under section 501 (a) of the Internal Revenue Code SPI
qualifies for 501 (c) (3) (non-profit organization) status under section
509 (a) (1) and 170 (b) (1) (A) (vi). This means that donations made to SPI
and its supported projects should be tax deductible for the American donor.

\end{document}

