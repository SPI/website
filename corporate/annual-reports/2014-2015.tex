\documentclass[letterpaper]{report}
\usepackage{parskip}
\usepackage{hyperref}

\hypersetup{
  colorlinks   = true,
  urlcolor     = blue,
  linkcolor    = blue,
}

\begin{document}

\title{Software in the Public Interest, Inc.\\
2014-2015 Annual Report}
\date{July 10, 2015}

\maketitle

To the membership, board and friends of Software in the Public Interest, Inc:

As mandated by Article 8 of the SPI Bylaws, I respectfully submit this annual
report on the activities of Software in the Public Interest, Inc. and extend my
thanks to all of those who contributed to the mission of SPI in the past year.

  \emph{-- Martin Michlmayr, SPI Secretary}

\newpage

\tableofcontents

\newpage

\chapter{President's Welcome}
\label{sec:president}

SPI continues to focus on our core services, quietly and competently
supporting the activities of our associated projects.

While the organization remains healthy, it has become clear that
maintaining the current level of service to our existing projects is
reaching the limits of our ability with an all-volunteer staff.  For that
reason, the board and I are currently investigating alternative approaches
to paying for part-time assistance with routine activities.  It seems
likely that some action will result in the next year, which I hope will
enable SPI to continue to grow and provide superior services to our
associated projects well into the future.

I continue to lament long-standing issues with our bylaws, but still have
not made time to work on them.  This is not glamorous work, and these
issues have existed since before I first joined the organization's board,
but I do remain hopefully that we might eventually make time to produce and
approve updated bylaws that match our current operational model.

A huge thank-you to everyone, particularly our board and other key
volunteers, whose various contributions of time and attention over the last
year made continued SPI operations possible!

  \emph{-- Bdale Garbee, SPI President}

\chapter{Committee Reports}
\section{Membership Committee}

\subsection{Statistics}

At the time of writing (July 1) the current membership status is:

\begin{tabular}{ | c | r | }
\hline
NC Applicants Pending Email Approval	& 255\\
NC Members				& 512\\
Contrib Membership Applications		& 8\\
Contrib Members				& 518\\
Application Managers			& 9\\
\hline
\end{tabular}

On July 1, 2014 we had 504 contributing and 455 non-contributing members.
On July 1, 2015 there were 518 contributing members and 512 non-contributing
members.


\chapter{Board Report}
\section{Board Members}

Board members as of July 1, 2014:

\begin{itemize}
\item Bdale Garbee (President)
\item Joerg Jaspert (Vice President)
\item Jonathan McDowell (Secretary)
\item Michael Schultheiss (Treasurer)
\item Clint Adams
\item Robert Brockway
\item Joshua D. Drake
\item Martin Michlmayr
\item Martin Zobel-Helas
\end{itemize}

Board members as of June 30, 2015:

\begin{itemize}
\item Bdale Garbee (President)
\item Joerg Jaspert (Vice President)
\item Martin Michlmayr (Secretary)
\item Michael Schultheiss (Treasurer)
\item Robert Brockway
\item Joshua D. Drake
\item Jonathan McDowell
\item Gregers Petersen
\item Martin Zobel-Helas
\end{itemize}

Advisors to the board as of June 30, 2015:

\begin{itemize}
\item Software Freedom Law Center (SFLC), legal counsel
\item Neil McGovern, Debian Project representative
\item Robert Treat, PostgreSQL Project representative
\end{itemize}

\section{Board Changes}

Changes that occurred during the year:

\begin{itemize}
\item The terms for Clint Adams, Robert Brockway and Martin Michlmayr
expired in July 2014.  Robert and Martin sought, and obtained,
re-election.  We thank Clint for his work on the board.  Gregers
Petersen joined the board as part of the same election.
\item On August 14, 2014 the board voted to extend the term of the
current President, Vice President and Treasurer by a further year.
These were:
\begin{itemize}
\item President: Bdale Garbee
\item Vice President: Joerg Jaspert
\item Treasurer: Michael Schultheiss
\end{itemize}
The board appointed Martin Michlmayr as new Secretary.
\end{itemize}

\section{Elections}

One board membership election was conducted in July 2014.  There were 3
board seats up for election.  Nominations were received from Robert
Brockway, Steve Langasek, Ben Longbons, Martin Michlmayr, Gregers
Petersen, and Trevor Walkley.  Robert Brockway, Martin Michlmayr, and
Gregers Petersen were elected to the board.

\chapter{Treasury Report}

This report uses a cash-based method of accounting, recording donations when
deposited (not when the check was written or received by us) and recording
expenses when sent or scheduled for payment (not when incurred).

\section{Income Statement}

This covers the Period July 1, 2014 -- June 30, 2015

\begin{verbatim}
Income
  Ordinary Income
    0 A.D.                           2,041.42
    aptosid                             40.02
    Arch Linux                       6,103.91
    Chakra                              52.00
    DebConf 14                      39,541.62
    DebConf 15                      73,648.46
    Debian                          66,449.10
    FFmpeg                          11,100.00
    FFmpeg (OPW)                       201.00
    The FreedomBox Foundation           35.30
    freedesktop.org                    666.00
    Gallery                             55.00
    GNU TeXmacs                         65.00
    haskell.org                      9,281.00
    The HeliOS Project                 196.12
    Jenkins                         11,710.00
    LibreOffice                     50,512.10
    MinGW                            2,476.00
    Open Bioinformatics Foundation  11,791.99
    Open Voting Foundation               2.00
    OpenWrt                            745.02
    PostgreSQL                      33,079.93
    Privoxy                            107.00
    SPI General                     46,834.11
    SPI 5%                           3,532.76
    Swathanthra Malayalam Computing  5,416.30
    TideSDK                            353.00
    YafaRay                          1,155.00

  Total Ordinary Income            377,191.16
                                   ----------

  Interest Income
    Key Business Platinum MM Savings    32.57
    Chase BusinessClassic Checking       0.12
    Chase Bus Select HighYield Savings 297.50
    Fifth Third Business MM 128         25.13

  Total Interest Income                355.32
                                       ------

  Gross Income                     377,546.48
                                   ----------
 
Expenses
  Ordinary Expenses

  0 A.D.
    Hosting                            632.54
    Processing Fees                     64.41
    Software License Fees              601.83
    SPI 5%                              67.70

  Total 0 A.D. Expenses              1,366.48
                                     --------

  aptosid
    Processing Fees                      3.36
    SPI 5%                               1.98

  Total aptosid Expenses                 5.34
                                         ----


  Arch Linux
    Processing Fees                    348.64
    SPI 5%                             339.06

  Total Arch Linux Expenses            687.70
                                       ------

  Chakra
    Processing Fees                      4.18
    SPI 5%                               2.60

    Total Chakra Expenses                6.78
                                         ----

  DebConf 14
    Badge materials                    208.50
    Conference bags                  1,059.75
    Conference fees (PSU)           53,942.95
    Daytrip                          2,980.75
    Expense reimbursement            3,721.36
    Lanyards                           826.21
    Network expenses                   460.69
    Postage                             11.40
    Processing fees                    448.90
    Reception                        9,621.60
    Stickers                            69.00
    Transfer to DC.de eV             6,500.00
    T-shirts                         2,845.83
    Travel reimbursement            35,268.81
    Travel/video                     2,521.42
    Water bottles                       30.29

  Total DebConf 14 Expenses        120,517.46
                                   ----------

  DebConf 15
    Postage                              3.94
    Processing fees                     41.87
    Wire transfer fees                  40.00

  Total DebConf 15 Expenses             85.81
                                        -----

  Debian
    Debit card fees                      36.00
    Domain registration fees             61.68
    Expense reimbursement             1,662.60
    Gnuk tokens                         210.00
    Hard drive                          226.92
    Hardware                            635.10
    Hardware (UK)                     1,981.67
    Hardware warranty                 1,142.31
    Mini DebConf Bucharest              592.00
    OPW internships                   9,419.10
    Processing fees                     891.84
    SPI 5%                            1,565.01
    SSDs for UBC                      6,462.37
    Travel (DC14 GSoC)               13,023.92

  Total Debian Expenses              37,910.52
                                     ---------

  FFmpeg
    OPW internship                    6,092.12
    Processing fees                     469.24
    SPI 5%                              559.00

  Total FFmpeg Expenses               7,120.36
                                      --------

  FFmpeg (OPW)
    OPW internship                      157.88
    Processing fees                      10.67
    SPI 5%                               10.05

  Total FFmpeg (OPW) Expenses           178.60
                                        ------

  freedesktop.org
    NPES Membership                      250.00
    PDF Association membership           327.82
    Processing Fees                       30.01
    SPI 5%                                33.30
    System Administration                802.50

  Total freedesktop.org Expenses       1,443.63
                                       --------

  The FreedomBox Foundation
    Processing Fees                        3.52
    SPI 5%                                 1.75

  Total The FreedomBox Foundation Expenses 5.27
                                           ----

  Gallery
    Processing Fees                        1.94
    SPI 5%                                 2.75

  Total Gallery Expenses                   4.69
                                           ----

  GNU TeXmacs
    Processing Fees                        3.30
    SPI 5%                                 3.25

  Total GNU TeXmacs Expenses               6.55
                                           ----

  haskell.org
    Hosting fees                       1,562.60
    Processing Fees                       79.71
    SPI 5%                               114.05

  Total haskell.org Expenses           1,756.36
                                       --------

  The HeliOS Project
    Expense Reimbursement              1,512.50
    Processing Fees                        5.94
    SPI 5%                                 9.81

  Total The HeliOS Project Expenses    1,528.25
                                       --------

  Jenkins
    Processing Fees                      466.20
    SPI 5%                               585.50

  Total Jenkins Expenses               1,051.70
                                       --------

  LibreOffice
    Brazilian domain name registrations  860.86
    Donation to Software Freedom
      Conservancy                        200.00
    Processing Fees                      142.54
    SPI 5%                               178.19
    Travel Reimbursements             61,116.48

  Total LibreOffice Expenses          62,498.07
                                      ---------

  MinGW
    Processing Fees                      105.43
    SPI 5%                               123.80

  Total MinGW Expenses                   229.23
                                         ------

  Open Bioinformatics Foundation
    Domain name registration fees        106.49
    Hosting fees                       2,074.47
    Processing Fees                       40.35
    SPI 5%                               100.00
    Travel (GSoC)                        821.00

  Total Open Bioinformatics Foundation
    Expenses                           3,142.31
                                       --------

  Open Voting Foundation
    Expense Reimbursement                154.00
    Processing Fees                        0.86
    SPI 5%                                 0.15

  Total Open Voting Foundation Expenses  155.01
                                         ------

  OpenWrt
    Processing Fees                       33.53
    SPI 5%                                37.25
    Trademark registration               275.00

  Total OpenWrt Expenses                 345.78
                                         ------

  PostgreSQL
    Expense Reimbursement                201.50
    Poster art                           300.00
    Processing Fees                       66.04
    SPI 5%                               122.60
    Travel Reimbursements              5,446.00
    Travel Reimbursements (GSoC)       1,219.53

  Total PostgreSQL Expenses            7,355.67
                                       --------

  Privoxy
    Processing Fees                        5.37
    SPI 5%                                 5.35

  Total Privoxy Expenses                  10.72
                                          -----

  SPI
    Bank fees                             193.99
    Bookkeeping                         5,900.00
    Office supplies                       131.13
    PaySimple                             379.40
    PO Box renewal                         92.00
    Registered Agent fees                 402.00
    Transaction fees                    3,658.13

  Total SPI Expenses                   10,756.65
                                       ---------

  Swathanthra Malayalam Computing
    Conference Fees                     2,300.00
    GSoC Travel                         2,941.73
    Processing Fees                        19.00

  Total Swathanthra Malayalam Computing
    Expenses                            5,260.73
                                        --------

  TideSDK
    Hosting Fees                          530.00
    Processing Fees                        11.12
    SPI 5%                                 17.05

  Total TideSDK Expenses                  558.17
                                          ------

  YafaRay
    Blender subscriptions                 267.56
    Domain name renewal fees               72.65
    Processing Fees                        30.10
    SPI 5%                                 57.75

  Total YafaRay Expenses                  428.06
                                          ------

  Total Expenses                      264,415.90
                                      ----------

  Net Income                          113,130.58
                                      ==========
\end{verbatim}

\section{Balance Sheet}

\begin{verbatim}
Balance Sheet as of June 30, 2015

ASSETS
  Current Assets
     Chase Performance Business Checking               68,596.40
     Chase Business Select High Yield Savings         205,676.77
     Fifth Third Business Money Market 128            145,591.71
     Fifth Third Business Elite Checking (SPI)             50.00
     Fifth Third Business Elite Checking (Debian)          50.00
     KeyBank Basic Business Checking                    8,928.08
     Key Business Reward Checking                     231,602.30
     Key Business Platinum Money Market Savings       162,867.47
     Key Express Checking                               5,297.72
     Ameriprise Cash Mgmt Acct                         13,406.15
     Debian Debit Card                                    194.00

  Total Current Assets                                842,260.60


TOTAL ASSETS                                          842,260.60


LIABILITIES & EQUITY

  General and current liabilities                           0.00

  Equity
     Reserves held in trust
        0 A.D. Earmark                    34,541.90
        ankur.org.in                       2,811.13
        aptosid Earmark                      251.14
        Arch Linux Earmark                21,333.04
        Chakra                                45.22
        Debian Earmark                   226,330.49
        DebConf 14 Earmark                35,962.78
        DebConf 15 Earmark                73,562.65
        Drizzle                            6,333.99
        FFmpeg                             8,420.20
        FFmpeg (OPW)                          23.40
        Fluxbox                              995.00
        freedesktop.org Earmark           16,059.88
        FreedomBox Foundation Earmark     24,955.61
        Gallery Earmark                    8,339.51
        GNU TeXmacs Earmark                  381.99
        Haskell Earmark                   32,986.82
        Jenkins Earmark                   25,233.89
        LibreOffice Earmark               49,478.34
        madwifi-project.org Earmark        1,494.90
        MinGW Earmark                      3,772.42
        OpenVAS                               56.21
        OpenWrt                            3,837.03
        Open Bioinformatics Earmark       47,168.66
        Open Embedded                        181.65
        Open Voting Foundation Earmark         5.73
        OSUNIX                                 2.92
        Path64                                18.60
        Plan 9 Earmark                     6,500.00
        PostgreSQL Earmark                70,158.08
        Privoxy Earmark                      163.99
        Swathanthra Malayan Comp Earmark   2,483.62
        The HeliOS Project Earmark           201.39
        TideSDK Earmark                      353.99
        Tux4Kids Earmark                  16,277.50
        YafaRay Earmark                    5,390.92

     Total held in trust                              726,114.59

     General reserves                                 116,146.01

  Total Equity                                        842,260.60

TOTAL LIABILITIES & EQUITY                            842,260.60
\end{verbatim}

(all sums in US dollars)


\chapter{Member Project Reports}

\section{New Associated Projects}

We have continued to see a reasonable level of interest from projects who wish
to become associated with SPI.  Over the past year one project has had
a successful resolution proposed for it to be invited to come under the SPI
umbrella as an Associated Project.


\subsection{The Mana World}

The Mana World (TMW) is a serious effort to create an innovative free and
open source MMORPG (Massively Multiplayer Online Role-Playing Game).  TMW
uses 2D graphics and aims to create a large and diverse interactive world.
It is licensed under the GPL.

\section{Updates from Associated Projects}

\subsection{0 A.D.}

0 A.D. (pronounced ``zero ey-dee'') is a cross-platform, real-time strategy
(RTS) game of ancient warfare. It's a historically-based war/economy game,
in which the player must lead an ancient civilization, gather resources
from the map, and raise a military force to conquer enemy factions. 0 A.D.
is open source software licensed under the GPL, and its art and sound
assets are licensed under CC BY-SA. It is developed by Wildfire Games, a
global community of game developers.

Between July 1 2014 to June 30 2015, we put out two alpha releases, each
available for Windows, OS X, Linux, and BSD, including long awaited
features as triggers, nomad maps, fogging, units on walls, a technology
tree, and new Seleucid structures. We were also able to upgrade the
codebase to use new technologies like C++11, SpiderMonkey 31, and a new
auto-builder, thanks to the new server which is funded thanks to our
donations via SPI. Apart from that, many performance improvements were
included and recently the new path-finder was merged.

{\em Submitted by Aviv Sharon}

\subsection{Chakra}

Chakra is a GNU/Linux distribution with an emphasis on KDE and Qt
technologies that focuses on simplicity from a technical standpoint and
free software.  In the past twelve months the Chakra community has made
three releases; 2014.09, 2014.11 and 2015.03.  We have also created two
additional virtual machines on our server for our unofficial Italian- and
Spanish-speaking communities to host their websites.  Last but not least,
we also gained an additional sponsor; JetBrains s.r.o.

{\em Submitted by Hans Tovetjärn}

\subsection{Debian}

The last twelve months have been a busy one for Debian, with
preparations for Debian 8.0, codenamed Jessie, reaching its final stages,
leading to a release on April 25, 2015. Debian 8.0 released with two
new architectures and a whole host of updated packages.

Additionally, efforts were started to increase the reliability and
verifiability of packages in Debian with source-only uploads, reproducible
builds and the removal of GPG keys less than 2048 bits from the keyring.

With the wider community, Debian again participated in Google Summer of
Code and Outreachy, as well as five sprints and the annual DebConf
gathering~--this time held in Portland, USA.

{\em Submitted by Neil McGovern}

\subsection{FFmpeg}

FFmpeg is a complete, cross-platform solution to record, convert and stream
audio and video. It is used as the platform foundation of many projects
dealing with multimedia, both open source and proprietary, and used
extensively by several multimedia web-based multimedia conversion and
processing services.

In the last twelve months FFmpeg delivered three main releases (2.4, 2.5,
2.6) and several security updates of old releases. A complete list of
changes can be
\href{http://git.videolan.org/?p=ffmpeg.git;a=blob_plain;f=Changelog;hb=HEAD}{found
in the changelog}.

In the last year FFmpeg participated into several development programs,
including OPW (now Outreachy) and Google Summer of Code.

{\em Submitted by Stefano Sabatini}

\subsection{FreedomBox}

FreedomBox Foundation has made substantial progress in the building of both
its educational program and technical demonstration materials in the past
year.  The FreedomBox server software stack will be having its 0.4 release
shortly.  We expect our demonstration software stack to be publicly shown
in a fully functional product line in late 2015 or early 2016.

The Foundation is also developing new materials for privacy education
around free technology; we will be announcing new initiatives in this
direction through 2015.

The Foundation plans to seek independent 501(c)(3) determination in the
near future.

{\em Submitted by Eben Moglen}

\subsection{Haskell.org}

Haskell is a programming language that supports strongly-typed, pure
functional programming.  In the last twelve months, the language marked an
important milestone with the release of GHC 7.10.  This release implemented
the long-awaited Applicative/Monad and Foldable/Traversable Proposals,
among with many other improvements to the standard libraries and language
core.

The last year has also seen the Haskell.org infrastructure greatly
improved, including a total website redesign and the completion of a
project to move web and other hosting to new, more reliable hardware.
These improvements have made many critical components of the shared
community infrastructure more user friendly and robust.

The Haskell community has continued to grow, in both the professional and
enthusiast spheres.  This year, Haskell.org is overseeing its largest
Google Summer of Code cohort yet, with 18 students participating from all
over the world.

{\em Submitted by Ryan Trinkle}

\subsection{LibreOffice}

LibreOffice follows a time-based release model to the benefit of not only
our end-users, but also our developers. New features are released to the
public in due time, and improvements are made available on a regular basis.
In 2014, The Document Foundation announced two major releases of
LibreOffice---LibreOffice 4.2 on January 30 and LibreOffice 4.3 on July 30.
15 minor releases have been made available as well. The LibreOffice Impress
Remote for iPhone and iPad was announced on March 2nd, for a total of 18
announcements in 12 months (on average, one announcement every 2.8 weeks or
20 days, which represents a significant achievement for a community-based
project). Developers started working on LibreOffice 4.4, and QA volunteers
organized two bug hunting sessions: the first in November after the release
of the first beta, and the second in December after the availability of the
first release candidate.

In addition to LibreOffice, there have been several announcements of
related products such as LibreOffice Portable, which allows to run
LibreOffice from USB key, and the LibreOffice viewer for Android. Third
parties have also announced solutions based on LibreOffice such as CloudOn
with the iPad editor, and RollApp with the virtualization technology which
brings LibreOffice on iOS, Android, Chrome OS and now Firefox OS. At the
end of 2014, the estimated user base of LibreOffice is exceeding 80 million
users worldwide according to the number of Windows and OS X users pinging
for updates, plus Linux users updating their software from repositories.

For more information on LibreOffice and The Document Foundation, see the
full
\href{https://wiki.documentfoundation.org/File:TDF2014AnnualReport.pdf}{2014
annual report}.

{\em Submitted by Sophie Gautier}

\subsection{PostgreSQL}

Over the last year PostgreSQL has continued to support outreach within our
community by sponsoring multiple speakers to multiple events. We
successfully held 4 major conferences: PostgresOpen, PgConf.eu, PgConf.US
and PgCon. We also had multiple single day events and our community
continues to grow through user groups and partnered events such as the
Southern California Linux Expo where the PostgreSQL track was popular
enough that the organizers had to move our room to give us more seats.

We continue to see growth within the community. This can be observed by
mailing list participation, IRC channel activity and other external sources
such as Stack Overflow. There is also anecdotal evidence that it has never
been a better time to be a PostgreSQL Consultant.

Lastly, this year we have had some growing pains with a number of high
profile bugs and releases happening in succession. This has been a source
of high activity as PostgreSQL is known for its reliability and the
community has been very vocal about keeping that reputation.

{\em Submitted by Joshua D. Drake}

\subsection{Privoxy}

Over the last twelve months the Privoxy project published two new stable
releases. Privoxy 3.0.22 included a couple of new features like processing
page content with scripts and programs written in any language (supported
by the OS).  Privoxy 3.0.23 was a bug fix release.

Donations were used to partially cover expenses for a conference visit and
parts of a laptop that is used for Privoxy development.

Compared to previous years development has slowed down a bit.  Due to
various issues with the hosting service a fair amount of developer time was
spent on finding an alternative that would allow Privoxy developers to
concentrate on development again, so far without success.

{\em Submitted by Fabian Keil}

\subsection{SproutCore}

SproutCore is a web development framework for creating complex web
applications that need to be fast, reliable and maintainable.  In the last
year we have delivered one bug fixing release (1.10.3) and one major
release (1.11.0).  We made substantial progress on replacing the current
Ruby-based build tools with one based on Node.js and SproutCore itself.
Much work has also been done to make the framework fully compatible with
ES5 strict mode.  See
\href{https://github.com/sproutcore/sproutcore/blob/master/CHANGELOG.md}{the
changelog} for more information.

{\em Submitted by Maurits Lamers}


\appendix
\chapter{About SPI}

SPI is a non-profit organization which was founded to help organizations
develop and distribute open hardware and software. We encourage programmers
to use the GNU General Public License or other licenses that allow free
redistribution and use of software, and hardware developers to distribute
documentation that will allow device drivers to be written for their product.

SPI was incorporated as a non-profit organization on June 16, 1997 in the state
of New York. Since then, it has become an umbrella organization for projects
from the community.

In 1999, the Internal Revenue Service (IRS) of the United States government
determined that under section 501 (a) of the Internal Revenue Code SPI
qualifies for 501 (c) (3) (non-profit organization) status under section 509
(a) (1) and 170 (b) (1) (A) (vi). This means that donations made to SPI and its
supported projects should be tax deductible for the American donor.

\end{document}
% Keep this at the bottom, thanks.
% Local Variables:
% TeX-master: "report"
% End:
