\documentclass[letterpaper]{report}
\usepackage{hyperref}

\begin{document}

\title{Software in the Public Interest, Inc.\\
2014 Annual Report}
\date{1st July 2014}

\maketitle

To the membership, board and friends of Software in the Public Interest, Inc:

As mandated by Article 8 of the SPI Bylaws, I respectfully submit this annual
report on the activities of Software in the Public Interest, Inc. and extend my
thanks to all of those who contributed to the mission of SPI in the past year.

  \emph{-- Jonathan McDowell, SPI Secretary}

\newpage

\tableofcontents

\newpage

\chapter{President's Welcome}
\label{sec:president}

  \emph{-- Bdale Garbee, SPI President}

\chapter{Committee Reports}
\section{Membership Committee}

\subsection{Statistics}

\chapter{Board Report}
\section{Board Members}

Board members as of July 1st, 2013:

\begin{itemize}
\item Bdale Garbee (President)
\item Joerg Jaspert (Vice President)
\item Jonathan McDowell (Secretary)
\item Michael Schultheiss (Treasurer)
\item Clint Adams
\item Robert Brockway
\item Joshua D. Drake
\item Jimmy Kaplowitz
\item Martin Zobel-Helas
\end{itemize}

Board members as of June 30th, 2014:

\begin{itemize}
\item Bdale Garbee (President)
\item Joerg Jaspert (Vice President)
\item Jonathan McDowell (Secretary)
\item Michael Schultheiss (Treasurer)
\item Clint Adams
\item Robert Brockway
\item Joshua D. Drake
\item Martin Michlmayr
\item Martin Zobel-Helas
\end{itemize}

Advisors to the board as of June 30th, 2014:

\begin{itemize}
\item Gregory Pomerantz, legal counsel
\item Software Freedom Law Center (SFLC), legal counsel
\item Lucas Nussbaum, Debian Project representative
\item Robert Treat, PostgreSQL Project representative
\end{itemize}

\section{Board Changes}

Changes that occurred during the year:

\begin{itemize}
\item The terms for Joshua D. Drake, Bdale Garbee, Joerg Jaspert and
Martin Zobel-Helas expired in July 2013. All of them sought, and
obtained, re-election.
\item On August 8th, 2013 the board voted to extend the term of the current
officers by a further year. These were:
\begin{itemize}
\item President: Bdale Garbee
\item Vice President: Joerg Jaspert
\item Secretary: Jonathan McDowell
\item Treasurer: Michael Schultheiss
\end{itemize}
\item On March 30th, 2014 Jimmy Kaplowitz expressed his intent to resign
due to lack of time and encouraged the board to appoint an interim
director.  On May 8th, 2014 the board appointed Martin Michlmayr as
interim director until the conclusion of the July 2014 board elections.
\end{itemize}

\section{Elections}

One board membership election was conducted in July 2013. There were 4
board members up for re-election.  Nominations were received from Joshua D.
Drake, Bdale Garbee, Joerg Jaspert, Gregers Petersen and Martin
Zobel-Helas.  Joshua D. Drake, Bdale Garbee, Joerg Jaspert and Martin
Zobel-Helas were elected to the board.

\chapter{Treasury Report}

This report uses a cash-based method of accounting, recording donations when
deposited (not when the check was written or received by us) and recording
expenses when sent or scheduled for payment (not when incurred).

\section{Income Statement}

This covers the Period July 01, 2013 - June 30, 2014

\begin{verbatim}
 Income
\end{verbatim}

\section{Balance Sheet}

\begin{verbatim}
As of June 30, 2014
\end{verbatim}

(all sums in US dollars)


\chapter{Member Project Reports}

\section{New Associated Projects}

We have continued to see a reasonable level of interest from projects who wish
to become associated with SPI. Over the past year 2 projects have had
successful resolutions proposed for them to be invited to come under the SPI
umbrella as an Associated Project.


\subsection{Chakra}

Chakra is a community based Linux distribution intended to provide a full
operating system primarily for desktop users.  Chakra fosters, promotes and
increases access to software systems available to the general public and
supports, encourages and promotes the creation and development of software
available to the general public.


\subsection{Swathanthra Malayalam Computing}

Swathanthra Malayalam Computing (SMC) is a free software project engaged
in development, localization, standardization and popularization of
various Free and Open Source Software in Indian Languages with a special
emphasis on Malayalam Language. SMC has been active since October 2002 and
has been working to provide various language tools that work on all layers
of computing including and: not limited to rendering fixes, fonts, input
mechanisms, translations (localization), text-to-speech engines,
dictionaries, spell checkers and other indic script based language
computing specific tools across operating systems. SMC also maintains
projects such as a generic web based Indic language computation framework
called SILPA, Dhvani TTS etc.


\appendix
\chapter{About SPI}

SPI is a non-profit organization which was founded to help organizations
develop and distribute open hardware and software. We encourage programmers
to use the GNU General Public License or other licenses that allow free
redistribution and use of software, and hardware developers to distribute
documentation that will allow device drivers to be written for their product.

SPI was incorporated as a non-profit organization on June 16, 1997 in the state
of New York. Since then, it has become an umbrella organization for projects
from the community.

In 1999, the Internal Revenue Service (IRS) of the United States government
determined that under section 501 (a) of the Internal Revenue Code SPI
qualifies for 501 (c) (3) (non- profit organization) status under section 509
(a) (1) and 170 (b) (1) (A) (vi). This means that donations made to SPI and its
supported projects should be tax deductible for the American donor.

\end{document}
% Keep this at the bottom, thanks.
% Local Variables:
% TeX-master: "report"
% End:
